\section{Electricity Contracts and Business Model}
\label{ElectrContr}

	The business environment here discussed involves the trading of three types of electricity arrangements: forward contracts with physical delivery obligation (supply contracts), capacity payment with renewable units and call options. It is assumed an ETC selling a supply contract to serve a firm load backed on renewable portfolio with the possibility to hedge the trading with call options. Throughout this work, we assume available a probability space $(\Omega, \mathcal{F}, \mathbb{P})$ with discrete sample space (plausible scenarios) where the random variables are defined \cite{Birge_StochProgm}. Additionally, for didactic purposes, we represent a random variable with an upper tilde (e.g. $\accentset{\sim}{\pi}$) and its realization in scenario $\omega \in \Omega$ as $\pi_{\omega}$. In the next subsections, we individually discuss each modality of contract and the role they play in this work. 

\subsection{Supply Contracts Backed on Renewable Generation}
\label{SupplyContract}

	Standard supply contracts are bilateral arrangements typically negotiated in electricity markets. They consist of an agreement to interchange a certain amount of energy $Q^{\text{sell}}$ (avgMW) during a specified time range at a price $P^{\text{sell}}$ (\$/MWh) \cite{RobustSpotPrice}. Specifically, the seller counterpart assumes full obligation to deliver the agreed energy in exchange of a fixed payment made by the buyer. Note that in this arrangement, the short-term risk is assumed by the seller since it has to buy in spot market the necessary energy to fulfill the contract. 

	A capacity (payment) contract is a modality of bilateral agreement which comprises energy coverage associated to a power source. Generally speaking, the buyer counterpart pays a fixed value to the underlying unit for the right to negotiate in the short-term market a percentage of its production, as a ``rent'' operation. In particular, when the unit involved is a renewable plant, the contract can be parametrized in two main variables: (i) a (fixed) price $P_{i}^{\text{res}}$ (\$/MWh) payed to the renewable generator for the respective (stochastic) energy production and (ii) a measure of ``firm energy'' $F_{i}^{\text{res}}$ (avgMW) associated to the unit. The idea behind a firm energy measure is to express the probabilistic nature of renewable production into a single value in order to quantify the amount of energy the producer is able to negotiate in the market \cite{SupplyAdequacy}. In this arrangement, the risk of short-term market exposure is transfered from the renewable producer to the contract buyer since the former receives a certain payment for its ``firm energy'' and the latter gains the right to negotiate in the short-term market the stochastic production of the renewable plant.

	Combining both contracts, the stochastic cash-flow of an ETC selling a supply contract backed by a set $U$ of renewable generators is presented next \cite{RobustSpotPrice}.
%
\begin{align}
	& R^{\text{supply}}_{t}(x^{\text{sell}}, \boldsymbol{x}^{\text{res}}, \accentset{\sim}{\pi}_{t}) = P^{\text{sell}} h_{t} Q^{\text{sell}} x^{\text{sell}} - \sum_{i \in U} P_{i}^{\text{res}} h_{t} F_{i}^{\text{res}} x_{i}^{\text{res}} \notag \\
	 		& \> \> \> \> \> \> \> \> \> \> + \Bigg( \sum_{i \in U} \accentset{\sim}{G}_{i,t}^{\text{res}} x_{i}^{\text{res}} - h_{t} Q^{\text{sell}} x^{\text{sell}} \Bigg)\accentset{\sim}{\pi}_{t}; \> \> \> \> \> \> \> \forall ~ t \in H, \label{RevSupplyCap}
\end{align}
%
where the first term represents a fixed supply payment received by the ETC, the second term indicates the fixed expenditure of each renewable generator in the portfolio and the third term stands for the short-term market settlement with $\accentset{\sim}{G}_{i,t}^{\text{res}}$ (MWh) and $\accentset{\sim}{\pi}_{t}$ (\$/MWh) the random variables representing the renewable production of the unit $i \in U$ and the energy spot price, respectively; $H$ defines the set of periods of the supply contract and $h_{t}$ the number of hours in period $t \in H$. In expression (\ref{RevSupplyCap}), $x^{\text{sell}}$ and $x_{i}^{\text{res}}$ represent, respectively, the percentage of energy sold in the supply contract and the percentage of each renewable source $i \in U$ brought to the portfolio. Note that this equation precisely reveals the price and volume risk faced by the ETC. For a given portfolio of renewable plants, if the ETC decides to fully sell the bilateral contract $(x^{\text{sell}} = 1)$, the fixed supply payment (first term of expression (\ref{RevSupplyCap})) is the highest possible. However, the energy unbalance exposure (third term of expression (\ref{RevSupplyCap})) and consequently the risk of purchase energy at high price levels in short-term market is also the highest possible. On the other hand, the decision to not sell the supply contract $(x^{\text{sell}} = 0)$ implies full renewable settlement in the short-term market, which is highly volatile \cite{SupplyAdequacy}.

\subsection{Call Options and the Business Revenue}

	With the intention of mitigating the price and volume risk discussed in section \ref{SupplyContract} we consider the possibility for the ETC to acquire European call options to hedge its portfolio. In the context of this work, a call option is a purely financial bilateral contract that gives the buyer (known as \textit{holder} of the option) the right, but not the obligation, to buy a certain amount of energy $Q_{j}^{\text{call}}$ (avgMW) at a specific time period for a fixed price $\Gamma_{j}^{\text{call}}$ (\$/MWh), called \textit{strike price}, defined \textit{a priori}. In exchange, the holder pays a fixed amount $P_{j}^{\text{call}}$ (\$/MWh), called option's \textit{premium}, to the contract seller (known as \textit{writer} of the option). The key point that makes energy call options a suitable hedge instrument against price and volume risk is the possibility to buy the agreed amount of energy at a known fixed price $\Gamma_{j}^{\text{call}}$. In other words, for scenarios $\omega \in \Omega$ of energy deficit and spot prices higher than the option's strike price, the buyer ``exercise'' the right to buy the agreed amount of energy at a fixed price level $\Gamma_{j}^{\text{call}} < \pi_{t,\omega}$ resulting in a payoff of $(\pi_{t,\omega} - \Gamma_{j}^{\text{call}})Q_{j}^{\text{call}} > 0$. On the other hand, for scenarios of spot prices below the strike price, there is no advantage in exercising the option, since energy can be bought at (relatively) low prices in the short-term market. Therefore, in the context of price and volume risk, call options are well-suited hedge instrument since they can efficiently cover the energy unbalance risk. Mathematically, for a given set $C_{t}$ of available call options in period $t \in H$, the stochastic cash-flow of a call option $j \in C_{t}$ is:
%
\begin{align}
	\hspace{-0.20cm} R_{j}^{\text{call}}(x_{j}^{\text{call}}, \accentset{\sim}{\pi}_{t}) = \Big( \text{max}\{0, \accentset{\sim}{\pi}_{t} - \Gamma_{j}^{\text{call}}\} - P_{j}^{\text{call}} \Big) h_{t} Q_{j}^{\text{call}} x_{j}^{\text{call}}. \label{CallCashFlow}
\end{align}

	The first term of (\ref{CallCashFlow}) reproduces the exercise rule discussed previously and the second term accounts for the premium paid to the option writer for the right embedded in the contract; $x_{j}^{\text{call}}$ defines the percentage of the $j \in C_{t}$ call option bought to hedge the portfolio. Expression (\ref{BusinessRevenue}) illustrates the resulting net revenue of an ETC during the whole business period by considering call options in the scheme described in section \ref{SupplyContract}.
%
\begin{align}
	& \hspace{-0.5cm} R(\mathbf{x}, \accentset{\sim}{\boldsymbol{\pi}}) = \sum_{t \in H} \Bigg[ P^{\text{sell}} h_{t} Q^{\text{sell}} x^{\text{sell}} - \sum_{i \in U} P_{i}^{\text{res}} h_{t} F_{i}^{\text{res}} x_{i}^{\text{res}} \notag \\
	 		& \> \> \> \> \> + \sum_{j \in C_{t}} \Big( \text{max}\{0, \accentset{\sim}{\pi}_{t} - \Gamma_{j}^{\text{call}}\} - P_{j}^{\text{call}} \Big) h_{t} Q_{j}^{\text{call}} x_{j}^{\text{call}} \notag \\
			& \> \> \> \> \> + \Bigg( \sum_{i \in U} \accentset{\sim}{G}_{i,t}^{\text{res}} x_{i}^{\text{res}} - h_{t} Q^{\text{sell}} x^{\text{sell}} \Bigg) \accentset{\sim}{\pi}_{t} \Bigg], \label{BusinessRevenue}
\end{align}
%
where $\mathbf{x} = [x^{\text{sell}}, \boldsymbol{x}^{\text{res}}, \boldsymbol{x}^{\text{call}}]^{\top}$. It is important to mention that the optimal composition of contracts should take into account the different characteristics and dynamics that each modality of contract holds and the synergy existent among them. For instance, the ETC must be aware of the tradeoff between the fixed payments (first and second terms of (\ref{BusinessRevenue})) and the price and volume risk introduced by the last term of (\ref{BusinessRevenue}) as well as the benefits of a portfolio of call options, balancing the cost incurred by the option's premium and the benefits in mitigating this risk. In the next section, we present the uncertainty characterization and the optimal portfolio allocation model proposed in this work.

%% Retirado temporariamente do texto


\subsection{Simulation Study}
\label{sec:simulation-ar1}
If we knew an autoregressive process true model
\[
y_t = \phi_0 + \sum_{p=1}^{P} \phi_p y_{t-p} + \varepsilon_t,
\]
and knew the error distribution, we could estimate the one-step ahead quantile as the sum $\hat{y}_{t+1} + t_{\alpha}$, where $t_\alpha$ is the $\alpha$-quantile for the error distribution. This means that we would be able to use all developments made on conditional mean estimation and simply add an error to estimate quantiles. 

However, that is not the case. When working with quantile estimation for real data we don't know the generating process exactly. Using a quantile regression model provides us a good solution even without making assumptions about the error distribution. 

We propose simulating an AR(1) model
\begin{equation}
y_t = \phi_0 +  \phi y_{t-1} + \varepsilon_t,\qquad \varepsilon_t \sim N(0, \sigma_\varepsilon^2),	
\label{eq:sim-true-model}
\end{equation}
and test two approaches to predict the one-step ahead quantile. On the first one, we consider known the process true model, given on equation \ref{eq:sim-true-model}. Thus, our task is to estimate values for $\hat{\phi}_0$, $\hat{\phi}$ and $\hat{\sigma}_\epsilon^2$. In order to calculate the one-step ahead $\alpha$-quantile, we need to compute
\begin{equation}
\hat{q}^{AR}_\alpha(x_t) = \hat{y}_{t+1|t} + z_\alpha \hat{\sigma}_\varepsilon,
\end{equation}
where $\hat{y}_{t+1|t} = \hat{\phi}_0 + \hat{\phi} y_{t}$ stands for the one-step ahead conditional mean and $z_\alpha = F^{-1}(\alpha)$, where $F$ is the gaussian distribution function.

On the second approach, we fit a quantile regression by solving problem \ref{eq:qar-lp}. The solution of this optimization problem are coefficients $\hat{\beta}_0$ and $\hat{\beta}$. In order to find the one-step ahead $\alpha$-quantile, we use the following expression:
\begin{equation}
\hat{q}^{QR}_\alpha(x_t) = \hat{\beta}_0 + \hat{\beta} y_{t}.
\end{equation}
Note that in both approaches we have one intercept term ($z_\alpha \sigma_\varepsilon + {\phi}_0$ and $\beta_0$) and a coefficient for the first lag ($\phi$ and $\beta$). 

We generate data according to equation \ref{eq:sim-true-model}, with different values for $\phi$ (0.25, 0.5, 0.7 and 0.9) and different signal to noise ratios (0.01, 0.05, 0.1, 0.5, 1). We use the signal to noise ratio (RSN) to form the error variance such that $\sigma_e^2 = \dfrac{\phi_0}{1-\phi} \cdot RSN$. This experiment was run with samples of size $n=20000$. The first half was used to fit coefficients and the second half was used as testing set, on which forecasting was done.

Our first goal is then to evaluate the ability of these two approaches to predict the one-step ahead $\alpha$-quantile for a few selected $\alpha$'s. We define the model $m$ forecasting error $\varepsilon_t^m$ as the quantity 
\begin{equation}
\varepsilon_{t \alpha}^m = \hat{q}^{m}_\alpha(x_t) - q_\alpha(x_t) = \hat{q}^{m}_\alpha(x_t) - \phi_0 - \phi y_{t-1} - z_\alpha \sigma_\epsilon.
\end{equation}
We use the Root Mean Squared Error (RMSE) to evaluate forecasting performance, which is defined for the model $m$ as follows:
\begin{equation}
RMSE^m = \sqrt{ \frac{1}{n-1} \sum_{i=2}^{n} \left( \varepsilon_{t \alpha}^m \right)^2}
\end{equation}

On section \ref{sec:simulation-tables}, we show a comparison of RMSE for both approaches. To compare them, we compute their RMSE ratio
\begin{equation}
R^{Q/A} = \dfrac{RMSE^{QR}}{RMSE^{AR}}.
\end{equation}
If $R^{Q/A}$ is smaller than 1, it means that the quantile regression approach performed better in terms of RMSE than the conditional mean based on autoregressive models approach.

Once simulations are executed, we notice forecasting errors are similar for both approaches. The exception being when $\phi$ is large (0.9) and the noise is small ($RSN = 0.01$). In those cases, quantile regression performed on average 10\% better than when using the conditional mean approach.
We also provide tables showing differences for the estimated autoregressive and intercept coefficients on section \ref{sec:simulation-tables}.
\section{Introduction}
\label{Introduction}

	\IEEEPARstart{T}{he} introduction of competitiveness in generation sector of most power systems around the world altered the way that Energy Trading Companies (ETCs) act in energy markets. Investments in new power plants are no longer driven by load growth, but by the search of an adequate balance between expected future profit and risk exposure. Furthermore, short-term energy trading and contract portfolio management became a relevant routine in most ETC's everyday activities. In this framework, a key challenge in decentralized markets relies on the determination of a set of investments and optimal contracting/bidding strategies that creates maximum value for electricity agents, taking into account several sources of uncertainties inherent to the business, such as energy, gas and fuel prices, as well as equipment deficiency, regulatory changes, failure in transmission lines, among others. In addition, when renewable sources are involved, their uncertain production is also a risk factor that must be considered in the decision problem \cite{RiskConstPortSelect}.

	Since the beginning of the deregulation process, several works have discussed methods to manage uncertainty on energy trading. One of the approaches widely adopted by practitioners is the acquisition of hedging instruments to protect their cash flow against unexpected events. In practice, there are two instruments commonly negotiated: forward contracts and options. Forward contracts are known to provide adequate hedge for thermal units against spot price volatility, which induces the so-called \textit{price risk} \cite{HedgingQtdRisk_Oren}. Their controlled self-generation prevents energy purchase in short-term markets at undesirable levels to fulfill the delivery obligation. However, in the case of renewable units -- such as hydro and wind plants -- the probabilistic and seasonal nature of their physical production induces an additional risk for the electricity company, the \textit{volume risk} \cite{HedgingQtdRisk_Oren}, which happens whenever the (stochastic) volume of energy produced is not sufficient to meet contract obligations and the company must buy the difference on highly volatile short-term market. Therefore, in addition to forward contracts, ETCs acquire different types of options with the objective of mitigating the joint price and volume risk associated with scenarios of energy unbalance and high spot price. In this context, a major challenge that energy companies face on competitive markets lie in the definition of the optimal mix between renewable portfolio and hedge instruments to be contracted through a co-optimization model.

	The problem of optimal portfolio allocation of electricity contracts has been widely studied in different contexts and market structures. Typically, the mathematical modeling starts with the identification of the uncertainty factors that affect the portfolio, e.g. renewable production and energy spot price. Then, a probabilistic description of these uncertainties is defined under the form of joint distribution functions or econometric/time-series models. Finally, a stochastic optimization problem that specifies the optimal portfolio of contracts to be settled is constructed. To solve such optimization problem, practitioners usually sample a set of ``scenarios'' (possible realizations) for the uncertainty factors from the available probability description and solve an approximate stochastic programming problem \cite{Birge_StochProgm}. A key point on implementing such modeling in the context of energy trading is to assess an accurate description of the probabilistic nature of the short-term market price. It is recognized that in practice agents only have an imprecise estimation of the ``true'' underlying process that guide the future spot price behavior. In addition, in the presence of severe price movements, this modeling error is worsened, which ultimately lead to non recoverable financial losses. Given this high difficulty on uncertainty characterization, decision makers act under \textit{ambiguity} and intuitively adjust their preferences over stochastic cash flows toward the ones derived from portfolios obtained under adverse conditions. Most of these agents follow a max-min structure where the optimal portfolio is constructed with the worst-possible realization of the uncertain parameters for the decision maker objectives \cite{SoftRobustModel_UnderAmb}. 

	Therefore, the main objective of this work is to devise an unified risk and ambiguity-constrained electricity portfolio allocation model of renewable, forward and call option contracts. Specifically, from a commercial point-of-view, we extend the business model proposed in \cite{RobustSpotPrice} to include the possibility of purchase call options to adequately hedge a portfolio of renewable sources against the price and volume risk induced by the sale of a forward contract. On the other hand, from the point-of-view of decision structure and preference relation, we propose an extension of the classical risk-constrained portfolio selection model introduced by Markowitz \cite{PortSelect_Markovitz} and commonly used on energy markets \cite{RiskConstPortSelect} to include the decision maker aversion to uncertainty on the probability modeling. For this purpose, we adapt the hybrid stochastic/robust methodology developed in \cite{RobustSpotPrice, AmbiguityEnergySpotPrice} to construct a set of constraints that restricts the business profit to a pre-specified level under the worst possible realization of the short-term market price. As discussed in \cite{AmbiguityEnergySpotPrice}, this modeling approach possesses a one-to-one relation to a set of probability distributions of the energy spot price. Thus, the constraints imposed by the hybrid stochastic/robust methodology, which we call \textit{ambiguity constraints}, performs a worst-case analysis over a set of probability distributions. Such approach is widely used in ambiguity theory \cite{SoftRobustModel_UnderAmb} to handle uncertainty on probability modeling. It is noteworthy that, although the presence of call options in the portfolio helps to mitigate the price and volume risk faced by the ETC, their proper value is highly dependent on the accuracy of the short-term price probability description. Therefore, in the context of modeling uncertainty, the classical risk-constrained allocation approach widely used to compose an ``optimal'' portfolio may provide a significant suboptimal portfolio, exposing the agent to unexpected financial losses. Thus, we argue that the proposed approach, which devises a risk and ambiguity-averse portfolio, provides a significant improvement on assessing the benefits of call options in price and quantity risk mitigation.

\subsection{Contributions Regarding the Existing Literature}

	Several works appear in recent technical literature that tackle the challenges of the optimal trading of energy in deregulated markets. Here, we provide close related works of the many references in energy trading in order to contextualize the contributions of this paper. In \cite{AStochDMFram_ElectrRetailer}, a stochastic portfolio optimization model for a retailer that defines the optimal involvement in supply sources to procure its demand is proposed. Also, \cite{HedgingQtdRisk_Oren} explores the typical correlation between energy spot price and load demand to derive an optimal zero-cost hedging function that maximizes the agent's expected utility and discusses a methodology to construct a portfolio of forward and option contracts that replicate the hedging payoff in a single-period setting. Similarly to the business structure considered in this work, \cite{ManagFinRiskWithCall} presents a three-stage stochastic optimization model to determine the optimal selling strategy, including options, forward contracts and pool trading, of a risk-averse price-taker thermal unit. In addition, \cite{RiskConstPortSelect} propose a risk-constrained allocation model to define the optimal mix of small hydro and biomass sources to back a supply contract in hydrothermal markets. From a robust optimization perspective, \cite{OfferingStrat_RO} provides a technique to co-optimize the self-scheduling and hourly offering curves of a price-taker producer in which the standard short-term market price forecast is replaced by confidence intervals. In \cite{AmbiguityEnergySpotPrice}, an interesting relation between a robust max-min model and ambiguity theory is provided with application to renewable trading. Similarly, \cite{RobustSpotPrice} adapts the robust optimization approach to perform an endogenous stress test for spot prices in order to devise a portfolio of renewable sources and supply contracts.

	Following the main ideas exposed on these previously reported works, we propose a portfolio allocation model that aims to devise an optimal portfolio of renewable sources, forward contracts and call options in a risk and ambiguity-constrained framework. We refer to \cite{SoftRobustModel_UnderAmb} for an extensive theoretical discussion of a similar decision structure proposed in this work with applications to financial asset allocation. From a methodological view-point, our model fits in the class of single-stage robust linear optimization problems \cite{PriceOfRobustness}, requiring thus only basic linear programming algorithms to solve \cite{Xpress}. It is worth emphasizing that to the best of our knowledge no work has been reported on portfolio allocation of renewable energy with forward and call options in a risk and ambiguity-constrained setting. Therefore, the contributions of this paper are threefold:

	\begin{enumerate}
		\item to extend the business model presented in \cite{RobustSpotPrice, AmbiguityEnergySpotPrice} to include the trading of call options;
		\item to handle the imprecision on future spot price probability characterization by means of ambiguity risk constraints;
		\item to provide an efficient solution methodology for the proposed multilevel portfolio selection problem.
	\end{enumerate}

%\subsection{Organization of This Work}

%	\textcolor{red}{This paper is structured as follows: in Section \ref{ElectrContr}, the electricity contracts involved in this work are discussed and the trading revenue is derived. Section \ref{OptContrStrat} presents the uncertainty modeling considered and the proposed portfolio allocation model. In Section \ref{SolMeth}, the proposed solution methodology for the portfolio problem is presented. In section \ref{CaseStudy}, we present case studies with realistic data of the Brazilian power system. Finally, in Section \ref{Conclusions} we outline the conclusions and future work.}
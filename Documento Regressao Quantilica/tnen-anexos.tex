\section{Appendices}
\label{appendice}

\subsection{Proof of quantiles as an optimization problem}
\label{sec:quantile-proof}
Let $Z^{\alpha}=\mbox{arg min}_{Q}E[\alpha\max\{0,X-Q\}+(1-\alpha)\max\{0,Q-X\}].$
We can rewrite the function as

\[
\begin{aligned}Y & =\alpha\int_{Q}^{\infty}(X-Q)dF_{x}+(1-\alpha)\int_{-\infty}^{Q}(Q-X)dF_{X}\\
& =\alpha\int_{Q}^{\infty}XdF_{x}-\alpha Q\int_{Q}^{\infty}QdF_{x}+Q\int_{-\infty}^{Q}dF_{x}-\int_{-\infty}^{Q}XdF_{x}-\alpha Q\int_{-\infty}^{Q}dF_{x}+\alpha\int_{-\infty}^{Q}XdF_{x}\\
& =\alpha\int_{Q}^{\infty}XdF_{x}-\alpha Q+QF_{X}(Q)-\int_{-\infty}^{Q}XdF_{x}-\alpha QF_{X}(Q)+\alpha\int_{-\infty}^{Q}XdF_{x}\\
& =\alpha\int_{Q}^{\infty}XdF_{x}-\alpha Q+QF_{X}(Q)-\int_{-\infty}^{Q}XdF_{x}+\alpha\int_{-\infty}^{Q}XdF_{x}
\end{aligned}
\]
By the first order condition for optimality, we need that $\frac{dZ(Q^{*})}{dQ}=0$.
So, we have:

\[
-\alpha Q^{*}f(Q^{*})-\alpha+F_{X}(Q^{*})+Q^{*}f(Q^{*})-Q^{*}f(Q^{*})+\alpha Q^{*}f(Q^{*})=0
\]
\[
F_{X}(Q^{*})=\alpha.
\]
Thus, we have that $Z^\alpha$ is the $\alpha-quantile$ of random variable $X$.

\subsection{MIP coefficients tables}

The following tables inform the size of Coefficients when using the regularization method based on MIP described on session \ref{sec:best-subset-mip}. When using this method, we choose a parameter $K$ which defines the total number of nonzero coefficients (without accounting the intercept $\beta_0$, which is always included). 
In each column we find the estimated values of coefficients for each different choice of $K$. As coefficients are quantile dependent, we provide tables for $\alpha \in (0.05, 0.1, 0.25, 0.5, 0.75, 0.9, 0.95)$.

%\caption{\textbf{$\alpha = 0.05.} 

% latex table generated in R 3.2.3 by xtable 1.7-4 package
% Sat Apr 30 15:33:56 2016
\begin{table}[ht]
\centering
\begin{tabular}{rrrrrrrrrrrrr}
  \hline
 & K=1 & K=2 & K=3 & K=4 & K=5 & K=6 & K=7 & K=8 & K=9 & K=10 & K=11 & K=12 \\ 
  \hline
$\beta_{0}$ & -15.33 & 9.38 & 1.48 & 1.34 & 8.72 & -1.68 & 4.94 & 0.65 & -0.27 & -0.16 & -3.96 & -2.55 \\ 
  $\beta_{1}$ & -0.00 & 0.79 & 0.66 & 0.58 & 0.46 & 0.40 & 0.48 & 0.46 & 0.46 & 0.47 & 0.42 & 0.44 \\ 
  $\beta_{2}$ & -0.00 & -0.00 & -0.00 & -0.00 & -0.00 & 0.33 & -0.00 & -0.00 & -0.00 & -0.00 & 0.14 & 0.09 \\ 
  $\beta_{3}$ & -0.00 & -0.00 & -0.00 & -0.00 & -0.00 & -0.00 & -0.00 & 0.20 & 0.20 & 0.19 & 0.20 & 0.17 \\ 
  $\beta_{4}$ & -0.00 & -0.47 & -0.28 & -0.27 & -0.29 & -0.35 & -0.31 & -0.40 & -0.35 & -0.35 & -0.34 & -0.31 \\ 
  $\beta_{5}$ & -0.00 & -0.00 & -0.00 & -0.00 & -0.00 & -0.00 & -0.00 & -0.00 & -0.00 & -0.05 & -0.07 & -0.09 \\ 
  $\beta_{6}$ & -0.00 & -0.00 & -0.00 & -0.00 & -0.00 & -0.00 & 0.11 & 0.08 & 0.11 & 0.17 & 0.12 & 0.19 \\ 
  $\beta_{7}$ & -0.00 & -0.00 & -0.00 & -0.00 & -0.00 & -0.00 & -0.00 & -0.00 & -0.16 & -0.15 & -0.08 & -0.15 \\ 
  $\beta_{8}$ & -0.00 & -0.00 & -0.00 & -0.00 & -0.15 & -0.00 & -0.31 & -0.26 & -0.17 & -0.17 & -0.16 & -0.18 \\ 
  $\beta_{9}$ & -0.00 & -0.00 & -0.00 & -0.00 & -0.00 & 0.14 & 0.16 & 0.20 & 0.26 & 0.23 & 0.28 & 0.33 \\ 
  $\beta_{10}$ & -0.00 & -0.00 & -0.00 & -0.00 & -0.00 & -0.00 & -0.00 & -0.00 & -0.00 & -0.00 & -0.00 & -0.04 \\ 
  $\beta_{11}$ & -0.00 & -0.00 & 0.26 & 0.17 & 0.21 & 0.08 & 0.16 & 0.19 & 0.17 & 0.18 & 0.17 & 0.20 \\ 
  $\beta_{12}$ & 1.17 & -0.00 & -0.00 & 0.18 & 0.15 & 0.19 & 0.22 & 0.20 & 0.20 & 0.18 & 0.18 & 0.17 \\ 
   \hline
\end{tabular}
\caption{Coefficients for quantile $\alpha = 0.05$}
\end{table}

% latex table generated in R 3.2.3 by xtable 1.8-2 package
% Wed Apr 20 16:50:47 2016
\begin{table}[ht]
\centering
\begin{tabular}{rrrrrrrrrrrrr}
  \hline
 & K=1 & K=2 & K=3 & K=4 & K=5 & K=6 & K=7 & K=8 & K=9 & K=10 & K=11 & K=12 \\ 
  \hline
beta\_0 & -10.68 & 10.07 & 3.56 & 1.24 & 0.76 & 3.01 & 3.33 & 3.02 & 1.05 & 2.26 & 1.55 & 1.57 \\ 
  beta\_1 & -0.00 & 0.81 & 0.63 & 0.61 & 0.55 & 0.49 & 0.49 & 0.50 & 0.48 & 0.44 & 0.44 & 0.44 \\ 
  beta\_2 & -0.00 & -0.00 & -0.00 & -0.00 & -0.00 & -0.00 & 0.04 & -0.00 & -0.00 & 0.04 & 0.07 & 0.07 \\ 
  beta\_3 & -0.00 & -0.00 & -0.00 & -0.00 & 0.15 & 0.20 & 0.16 & 0.15 & 0.13 & 0.11 & 0.12 & 0.12 \\ 
  beta\_4 & -0.00 & -0.43 & -0.33 & -0.28 & -0.37 & -0.33 & -0.34 & -0.30 & -0.24 & -0.24 & -0.26 & -0.25 \\ 
  beta\_5 & -0.00 & -0.00 & -0.00 & -0.00 & -0.00 & -0.08 & -0.07 & -0.12 & -0.14 & -0.15 & -0.17 & -0.17 \\ 
  beta\_6 & -0.00 & -0.00 & -0.00 & -0.00 & -0.00 & -0.00 & -0.00 & 0.11 & 0.10 & 0.10 & 0.14 & 0.14 \\ 
  beta\_7 & -0.00 & -0.00 & -0.00 & -0.00 & -0.00 & -0.00 & -0.00 & -0.07 & -0.11 & -0.13 & -0.11 & -0.11 \\ 
  beta\_8 & -0.00 & -0.00 & -0.00 & -0.00 & -0.00 & -0.00 & -0.00 & -0.00 & -0.00 & -0.00 & -0.04 & -0.04 \\ 
  beta\_9 & -0.00 & -0.00 & -0.00 & -0.00 & -0.00 & -0.00 & -0.00 & -0.00 & 0.09 & 0.10 & 0.13 & 0.13 \\ 
  beta\_10 & -0.00 & -0.00 & -0.00 & -0.00 & -0.00 & -0.00 & -0.00 & -0.00 & -0.00 & -0.00 & -0.00 & 0.00 \\ 
  beta\_11 & -0.00 & -0.00 & -0.00 & 0.14 & 0.17 & 0.17 & 0.16 & 0.15 & 0.11 & 0.09 & 0.08 & 0.08 \\ 
  beta\_12 & 1.09 & -0.00 & 0.35 & 0.27 & 0.25 & 0.22 & 0.22 & 0.26 & 0.33 & 0.34 & 0.33 & 0.33 \\ 
   \hline
\end{tabular}
\end{table}

% latex table generated in R 3.2.3 by xtable 1.7-4 package
% Mon Apr 18 18:43:05 2016
\begin{table}[ht]
\centering
\begin{tabular}{rrrrrrrrrrrrr}
  \hline
 & K=1 & K=2 & K=3 & K=4 & K=5 & K=6 & K=7 & K=8 & K=9 & K=10 & K=11 & K=12 \\ 
  \hline
beta\_0 & 2.72 & -3.38 & 8.64 & 4.88 & 0.62 & 2.98 & 2.70 & 2.62 & 2.27 & 1.87 & 2.43 & 2.53 \\ 
  beta\_1 & -0.00 & 0.59 & 0.52 & 0.51 & 0.57 & 0.54 & 0.56 & 0.56 & 0.58 & 0.58 & 0.57 & 0.57 \\ 
  beta\_2 & -0.00 & -0.00 & -0.00 & -0.00 & -0.00 & -0.00 & -0.00 & -0.00 & -0.03 & -0.06 & -0.05 & -0.05 \\ 
  beta\_3 & -0.00 & -0.00 & -0.00 & -0.00 & -0.00 & -0.00 & -0.00 & -0.00 & -0.00 & 0.04 & 0.03 & 0.04 \\ 
  beta\_4 & -0.00 & -0.00 & -0.25 & -0.18 & -0.14 & -0.11 & -0.11 & -0.12 & -0.11 & -0.11 & -0.11 & -0.12 \\ 
  beta\_5 & -0.00 & -0.00 & -0.00 & -0.00 & -0.00 & -0.00 & -0.00 & -0.00 & -0.00 & -0.00 & -0.00 & 0.01 \\ 
  beta\_6 & -0.00 & -0.00 & -0.00 & -0.00 & -0.00 & -0.06 & -0.09 & -0.08 & -0.08 & -0.08 & -0.09 & -0.09 \\ 
  beta\_7 & -0.00 & -0.00 & -0.00 & -0.00 & -0.00 & -0.00 & -0.00 & -0.00 & -0.00 & -0.00 & -0.02 & -0.02 \\ 
  beta\_8 & -0.00 & -0.00 & -0.00 & -0.00 & -0.00 & -0.00 & 0.06 & 0.06 & 0.05 & 0.06 & 0.08 & 0.07 \\ 
  beta\_9 & -0.00 & -0.00 & -0.00 & -0.00 & 0.08 & 0.09 & 0.06 & 0.09 & 0.07 & 0.07 & 0.08 & 0.08 \\ 
  beta\_10 & -0.00 & -0.00 & -0.00 & -0.00 & -0.00 & -0.00 & -0.00 & -0.05 & -0.04 & -0.05 & -0.05 & -0.05 \\ 
  beta\_11 & -0.00 & 0.54 & -0.00 & 0.15 & 0.14 & 0.11 & 0.10 & 0.11 & 0.14 & 0.14 & 0.15 & 0.14 \\ 
  beta\_12 & 0.92 & -0.00 & 0.42 & 0.34 & 0.32 & 0.33 & 0.32 & 0.34 & 0.33 & 0.34 & 0.32 & 0.33 \\ 
   \hline
\end{tabular}
\end{table}

% latex table generated in R 3.2.3 by xtable 1.7-4 package
% Mon Apr 18 18:43:05 2016
\begin{table}[ht]
\centering
\begin{tabular}{rrrrrrrrrrrrr}
  \hline
 & K=1 & K=2 & K=3 & K=4 & K=5 & K=6 & K=7 & K=8 & K=9 & K=10 & K=11 & K=12 \\ 
  \hline
beta\_0 & 12.14 & 10.06 & 6.60 & 11.05 & 13.22 & 12.04 & 13.34 & 13.28 & 12.58 & 13.69 & 13.47 & 13.71 \\ 
  beta\_1 & -0.00 & 0.24 & 0.39 & 0.39 & 0.40 & 0.38 & 0.38 & 0.38 & 0.38 & 0.40 & 0.40 & 0.40 \\ 
  beta\_2 & -0.00 & -0.00 & -0.00 & -0.00 & -0.00 & -0.00 & -0.00 & -0.00 & -0.00 & -0.00 & -0.00 & -0.02 \\ 
  beta\_3 & -0.00 & -0.00 & -0.00 & -0.00 & -0.00 & -0.00 & -0.00 & -0.00 & -0.01 & -0.04 & -0.03 & -0.02 \\ 
  beta\_4 & -0.00 & -0.00 & -0.00 & -0.00 & -0.00 & -0.00 & -0.00 & 0.03 & -0.00 & 0.05 & 0.05 & 0.04 \\ 
  beta\_5 & -0.00 & -0.00 & -0.00 & -0.00 & -0.00 & -0.00 & -0.00 & -0.00 & -0.00 & -0.00 & 0.00 & 0.01 \\ 
  beta\_6 & -0.00 & -0.00 & -0.00 & -0.14 & -0.00 & -0.00 & -0.03 & -0.05 & -0.01 & -0.07 & -0.07 & -0.07 \\ 
  beta\_7 & -0.00 & -0.00 & -0.00 & -0.00 & -0.19 & -0.10 & -0.10 & -0.11 & -0.09 & -0.11 & -0.11 & -0.10 \\ 
  beta\_8 & -0.00 & -0.00 & -0.00 & -0.00 & -0.00 & -0.08 & -0.07 & -0.08 & -0.08 & -0.07 & -0.07 & -0.08 \\ 
  beta\_9 & -0.00 & -0.00 & -0.00 & 0.14 & 0.16 & 0.15 & 0.16 & 0.18 & 0.16 & 0.19 & 0.19 & 0.19 \\ 
  beta\_10 & -0.00 & -0.00 & -0.00 & -0.00 & -0.00 & -0.00 & -0.00 & -0.00 & -0.04 & -0.06 & -0.06 & -0.06 \\ 
  beta\_11 & -0.00 & -0.00 & 0.20 & -0.00 & 0.11 & 0.15 & 0.12 & 0.16 & 0.16 & 0.18 & 0.18 & 0.19 \\ 
  beta\_12 & 0.80 & 0.63 & 0.39 & 0.42 & 0.26 & 0.29 & 0.28 & 0.23 & 0.29 & 0.24 & 0.24 & 0.25 \\ 
   \hline
\end{tabular}
\end{table}

% latex table generated in R 3.2.3 by xtable 1.7-4 package
% Sat Apr 30 15:33:56 2016
\begin{table}[ht]
\centering
\begin{tabular}{rrrrrrrrrrrrr}
  \hline
 & K=1 & K=2 & K=3 & K=4 & K=5 & K=6 & K=7 & K=8 & K=9 & K=10 & K=11 & K=12 \\ 
  \hline
$\beta_{0}$ & 16.73 & 11.74 & 11.51 & 13.77 & 13.45 & 13.48 & 14.36 & 14.84 & 12.36 & 14.04 & 13.09 & 14.00 \\ 
  $\beta_{1}$ & -0.00 & 0.26 & 0.32 & 0.35 & 0.38 & 0.38 & 0.40 & 0.43 & 0.40 & 0.40 & 0.39 & 0.39 \\ 
  $\beta_{2}$ & -0.00 & -0.00 & -0.00 & -0.00 & -0.00 & -0.00 & -0.00 & -0.00 & -0.00 & -0.00 & 0.02 & 0.02 \\ 
  $\beta_{3}$ & -0.00 & -0.00 & -0.00 & -0.00 & -0.00 & -0.00 & -0.00 & -0.00 & -0.00 & -0.00 & -0.00 & 0.01 \\ 
  $\beta_{4}$ & -0.00 & -0.00 & -0.00 & -0.00 & -0.00 & -0.00 & -0.00 & -0.00 & 0.04 & 0.06 & 0.06 & 0.05 \\ 
  $\beta_{5}$ & -0.00 & -0.00 & -0.00 & -0.00 & -0.00 & -0.00 & -0.00 & -0.00 & -0.00 & -0.04 & -0.03 & -0.04 \\ 
  $\beta_{6}$ & -0.00 & -0.00 & -0.00 & -0.00 & -0.00 & -0.00 & -0.05 & -0.10 & -0.07 & -0.09 & -0.08 & -0.09 \\ 
  $\beta_{7}$ & -0.00 & -0.00 & -0.00 & -0.15 & -0.14 & -0.12 & -0.09 & -0.05 & -0.06 & -0.06 & -0.06 & -0.06 \\ 
  $\beta_{8}$ & -0.00 & -0.00 & -0.00 & -0.00 & -0.00 & -0.04 & -0.05 & -0.07 & -0.05 & -0.08 & -0.07 & -0.07 \\ 
  $\beta_{9}$ & -0.00 & -0.00 & -0.00 & 0.16 & 0.11 & 0.14 & 0.16 & 0.19 & 0.19 & 0.22 & 0.22 & 0.21 \\ 
  $\beta_{10}$ & -0.00 & -0.00 & -0.00 & -0.00 & -0.00 & -0.00 & -0.00 & -0.15 & -0.14 & -0.11 & -0.12 & -0.11 \\ 
  $\beta_{11}$ & -0.00 & -0.00 & 0.17 & -0.00 & 0.14 & 0.13 & 0.12 & 0.25 & 0.23 & 0.18 & 0.21 & 0.22 \\ 
  $\beta_{12}$ & 0.71 & 0.59 & 0.37 & 0.41 & 0.28 & 0.28 & 0.25 & 0.21 & 0.27 & 0.25 & 0.24 & 0.22 \\ 
   \hline
\end{tabular}
\end{table}

% latex table generated in R 3.3.2 by xtable 1.8-2 package
% Thu Feb  9 18:57:39 2017
\begin{table}[ht]
\centering
\begin{tabular}{rrrrrrrrrrrrr}
  \hline
 & Jan & Feb & Mar & Apr & May & Jun & Jul & Aug & Sep & Oct & Nov & Dec \\ 
  \hline
1981 & 23.36 & 28.34 & 12.44 & 18.35 & 17.10 & 22.49 & 23.57 & 40.10 & 48.40 & 42.13 & 43.70 & 37.23 \\ 
  1982 & 20.54 & 17.48 & 7.42 & 10.87 & 16.57 & 20.79 & 27.95 & 42.55 & 49.12 & 42.48 & 44.78 & 40.20 \\ 
  1983 & 27.94 & 24.50 & 22.60 & 22.24 & 29.62 & 27.05 & 33.92 & 45.06 & 50.64 & 49.32 & 43.83 & 36.14 \\ 
  1984 & 20.37 & 15.35 & 3.94 & 3.57 & 7.85 & 14.65 & 20.56 & 41.01 & 44.58 & 44.31 & 42.94 & 31.65 \\ 
  1985 & 10.38 & 4.71 & 5.15 & 2.84 & 7.27 & 10.36 & 14.53 & 39.33 & 45.18 & 41.21 & 42.15 & 23.02 \\ 
  1986 & 18.86 & 8.25 & 3.00 & 5.23 & 17.29 & 17.85 & 23.08 & 41.36 & 48.30 & 42.83 & 44.36 & 36.41 \\ 
  1987 & 26.09 & 24.71 & 6.90 & 21.02 & 20.73 & 19.53 & 28.42 & 42.94 & 48.06 & 44.26 & 43.11 & 39.67 \\ 
  1988 & 15.75 & 11.66 & 4.51 & 4.36 & 8.29 & 11.50 & 19.10 & 38.40 & 46.47 & 44.80 & 41.79 & 22.40 \\ 
  1989 & 19.92 & 14.52 & 5.08 & 2.75 & 5.62 & 11.42 & 17.17 & 38.94 & 43.92 & 43.70 & 40.69 & 26.34 \\ 
  1990 & 29.74 & 11.70 & 15.69 & 14.02 & 14.85 & 22.28 & 24.02 & 44.55 & 48.18 & 44.66 & 41.51 & 32.41 \\ 
  1991 & 17.09 & 13.46 & 7.68 & 6.63 & 8.51 & 16.17 & 26.46 & 43.36 & 49.00 & 45.86 & 40.14 & 36.57 \\ 
  1992 & 21.41 & 19.78 & 14.25 & 21.45 & 24.24 & 24.64 & 30.34 & 45.43 & 51.33 & 47.66 & 44.50 & 37.97 \\ 
  1993 & 27.86 & 20.13 & 14.36 & 16.63 & 20.94 & 26.43 & 30.60 & 44.07 & 44.73 & 43.78 & 41.40 & 34.18 \\ 
  1994 & 12.45 & 11.06 & 4.70 & 5.85 & 10.49 & 11.04 & 23.03 & 38.50 & 48.92 & 47.30 & 44.97 & 36.55 \\ 
  1995 & 20.31 & 5.80 & 9.47 & 5.36 & 5.62 & 14.15 & 23.54 & 42.48 & 50.49 & 42.74 & 41.15 & 29.90 \\ 
  1996 & 19.89 & 11.85 & 3.43 & 5.08 & 8.26 & 16.29 & 24.89 & 40.52 & 48.44 & 44.92 & 40.15 & 36.37 \\ 
  1997 & 23.89 & 27.80 & 14.30 & 11.95 & 17.55 & 22.22 & 31.82 & 44.07 & 43.14 & 40.00 & 37.94 & 28.36 \\ 
  1998 & 15.04 & 21.70 & 10.61 & 17.28 & 21.57 & 22.31 & 27.26 & 42.45 & 49.04 & 46.76 & 37.22 & 35.74 \\ 
  1999 & 22.18 & 15.39 & 8.18 & 13.66 & 8.67 & 16.49 & 22.30 & 40.43 & 47.75 & 39.85 & 36.95 & 35.54 \\ 
  2000 & 16.75 & 7.95 & 11.33 & 10.47 & 16.73 & 15.07 & 18.90 & 38.91 & 44.26 & 46.34 & 41.98 & 31.62 \\ 
  2001 & 24.03 & 11.82 & 11.09 & 9.23 & 16.30 & 14.53 & 25.73 & 41.57 & 45.79 & 40.99 & 41.52 & 42.76 \\ 
  2002 & 16.81 & 22.08 & 13.40 & 11.07 & 15.71 & 17.52 & 26.55 & 41.64 & 45.80 & 45.94 & 40.64 & 30.58 \\ 
  2003 & 17.42 & 14.05 & 10.03 & 11.26 & 15.39 & 17.01 & 28.29 & 39.98 & 47.02 & 47.07 & 40.47 & 34.85 \\ 
  2004 & 15.04 & 13.34 & 17.84 & 16.97 & 20.10 & 19.48 & 25.03 & 40.11 & 48.25 & 47.21 & 44.13 & 35.79 \\ 
  2005 & 24.89 & 20.47 & 13.01 & 20.88 & 19.98 & 21.48 & 27.81 & 42.74 & 46.09 & 46.93 & 44.98 & 36.08 \\ 
  2006 & 32.48 & 15.44 & 12.93 & 6.59 & 12.19 & 19.08 & 27.79 & 40.72 & 46.01 & 44.38 & 42.85 & 33.99 \\ 
  2007 & 28.93 & 11.13 & 16.10 & 11.91 & 17.68 & 21.57 & 30.56 & 42.95 & 47.80 & 47.61 & 42.97 & 35.98 \\ 
  2008 & 20.42 & 15.46 & 3.51 & 9.37 & 8.71 & 13.02 & 23.61 & 36.93 & 45.82 & 46.49 & 43.91 & 35.19 \\ 
  2009 & 21.48 & 15.16 & 6.74 & 3.80 & 4.48 & 12.88 & 24.53 & 38.40 & 47.70 & 40.87 & 46.73 & 38.03 \\ 
  2010 & 24.75 & 30.70 & 16.99 & 16.95 & 15.72 & 16.86 & 27.43 & 43.18 & 48.71 & 35.79 & 41.30 & 30.15 \\ 
  2011 & 16.33 & 14.79 & 9.30 & 7.70 & 13.35 & 18.60 & 23.53 & 39.62 & 46.97 & 40.99 & 44.75 & 42.79 \\ 
   \hline
\end{tabular}
\end{table}



\subsection{Simulation study tables}
\label{sec:simulation-tables}

On section \ref{sec:simulation-ar1} we explain a simulation study to try evaluating differences between estimating a quantile with a quantile regression model or using the conditional mean when knowing the true generating model. From this experiment, we present below table of results for three different aspects, for five different quantiles. Tables from \ref{tab:sim-rmse-005} to \ref{tab:sim-rmse-095}  shows the difference between the root mean square errors between both methods of predicting the one-step ahead quantile for a few different values of $\alpha$. Tables \ref{tab:sim-auto-005}-\ref{tab:sim-auto-095} are the ones that shows the nominal difference between the autoregressive coefficients $\hat{\phi}$ and $\hat{\beta}$. The nominal difference between the intercept terms $\hat{\phi}_0 + z_\alpha  \hat{\sigma}^2_\varepsilon)$ and $\hat{\beta}_0)$ is 

% latex table generated in R 3.2.3 by xtable 1.8-2 package
% Tue May 24 17:06:45 2016
\begin{table}[ht]
\centering
\begin{tabular}{rrrrrr}
  \hline
$\phi \backslash RSN$ & 0.01 & 0.05 & 0.1 & 0.5 & 1 \\ 
  \hline
0.25 & 0.99454499 & 0.99987753 & 0.99726238 & 0.99996553 & 0.99997820 \\ 
  0.50 & 0.99928262 & 1.00002733 & 0.99902661 & 1.00017501 & 1.00072601 \\ 
  0.70 & 0.97952103 & 0.99154652 & 1.00049301 & 0.99991315 & 1.00014586 \\ 
  0.90 & 0.89928729 & 0.99117750 & 0.99730087 & 1.00106737 & 0.99970135 \\ 
   \hline
\end{tabular}
\caption{RMSE ratio ($RMSE^{QR} / RMSE^{AR} $) for estimating quantile
$\alpha = $ 0.05. $\phi$ stands for the autoregressive coefficient 
and RSN is the signal to noise ratio. Details for these experiments can 
be found on section \ref{sec:simulation-ar1}} 
\label{tab:sim-rmse-005}
\end{table}

% latex table generated in R 3.2.3 by xtable 1.8-2 package
% Tue May 24 17:06:45 2016
\begin{table}[ht]
\centering
\begin{tabular}{rrrrrr}
  \hline
$\phi \backslash RSN$ & 0.01 & 0.05 & 0.1 & 0.5 & 1 \\ 
  \hline
0.25 & 0.99404378 & 1.00071379 & 0.99865043 & 1.00009650 & 1.00031150 \\ 
  0.50 & 0.99302385 & 1.00012893 & 1.00008177 & 1.00020172 & 0.99997679 \\ 
  0.70 & 0.96163465 & 0.99948879 & 1.00007350 & 1.00076714 & 1.00002219 \\ 
  0.90 & 0.82867689 & 0.99989835 & 1.00007245 & 0.99997754 & 1.00016876 \\ 
   \hline
\end{tabular}
\caption{RMSE ratio ($RMSE^{QR} / RMSE^{AR} $) for estimating quantile
$\alpha = $ 0.1. $\phi$ stands for the autoregressive coefficient 
and RSN is the signal to noise ratio. Details for these experiments can 
be found on section \ref{sec:simulation-ar1}} 
\label{tab:sim-rmse-01}
\end{table}

% latex table generated in R 3.2.3 by xtable 1.8-2 package
% Tue May 24 17:06:46 2016
\begin{table}[ht]
\centering
\begin{tabular}{rrrrrr}
  \hline
$\phi \backslash RSN$ & 0.01 & 0.05 & 0.1 & 0.5 & 1 \\ 
  \hline
0.25 & 0.98066380 & 0.99981708 & 0.99972368 & 1.00003875 & 1.00027661 \\ 
  0.50 & 0.99671712 & 0.99884992 & 0.99718531 & 0.99996169 & 0.99939155 \\ 
  0.70 & 0.96160330 & 0.99971110 & 1.00006113 & 1.00001116 & 1.00024519 \\ 
  0.90 & 0.93616426 & 0.99985323 & 0.99934405 & 0.99991411 & 0.99994730 \\ 
   \hline
\end{tabular}
\caption{RMSE ratio ($RMSE^{QR} / RMSE^{AR} $) for estimating quantile
$\alpha = $ 0.5. $\phi$ stands for the autoregressive coefficient 
and RSN is the signal to noise ratio. Details for these experiments can 
be found on section \ref{sec:simulation-ar1}} 
\label{tab:sim-rmse-05}
\end{table}

% latex table generated in R 3.2.3 by xtable 1.8-2 package
% Tue May 24 17:06:46 2016
\begin{table}[ht]
\centering
\begin{tabular}{rrrrrr}
  \hline
$\phi \backslash RSN$ & 0.01 & 0.05 & 0.1 & 0.5 & 1 \\ 
  \hline
0.25 & 0.99662610 & 0.99945043 & 0.99538478 & 0.99966478 & 1.00025526 \\ 
  0.50 & 1.00028533 & 0.99613341 & 0.99981279 & 0.99996292 & 1.00108295 \\ 
  0.70 & 0.91878126 & 0.99889801 & 0.99885007 & 1.00023457 & 1.00067966 \\ 
  0.90 & 0.80536841 & 0.99826964 & 0.99806239 & 1.00003275 & 0.99993464 \\ 
   \hline
\end{tabular}
\caption{RMSE ratio ($RMSE^{QR} / RMSE^{AR} $) for estimating quantile
$\alpha = $ 0.9. $\phi$ stands for the autoregressive coefficient 
and RSN is the signal to noise ratio. Details for these experiments can 
be found on section \ref{sec:simulation-ar1}} 
\label{tab:sim-rmse-09}
\end{table}

% latex table generated in R 3.2.3 by xtable 1.8-2 package
% Tue May 24 17:06:46 2016
\begin{table}[ht]
\centering
\begin{tabular}{rrrrrr}
  \hline
$\phi \backslash RSN$ & 0.01 & 0.05 & 0.1 & 0.5 & 1 \\ 
  \hline
0.25 & 1.00009586 & 0.99962843 & 1.00162321 & 1.00002799 & 1.00023607 \\ 
  0.50 & 0.99977885 & 0.99742853 & 0.99415437 & 1.00058173 & 0.99971703 \\ 
  0.70 & 0.93295124 & 0.99948015 & 0.99936096 & 1.00048504 & 0.99969376 \\ 
  0.90 & 0.97068193 & 1.00001858 & 1.00007662 & 1.00020540 & 0.99928392 \\ 
   \hline
\end{tabular}
\caption{RMSE ratio ($RMSE^{QR} / RMSE^{AR} $) for estimating quantile
$\alpha = $ 0.95. $\phi$ stands for the autoregressive coefficient 
and RSN is the signal to noise ratio. Details for these experiments can 
be found on section \ref{sec:simulation-ar1}} 
\label{tab:sim-rmse-095}
\end{table}


% latex table generated in R 3.2.3 by xtable 1.8-2 package
% Tue May 24 17:06:45 2016
\begin{table}[ht]
\centering
\begin{tabular}{rrrrrr}
  \hline
$\phi \backslash RSN$ & 0.01 & 0.05 & 0.1 & 0.5 & 1 \\ 
  \hline
0.25 & -0.02501316 & 0.01504215 & -0.00565925 & -0.01735387 & -0.00060544 \\ 
  0.50 & -0.01319163 & 0.00222293 & 0.00477565 & 0.01005289 & -0.01981673 \\ 
  0.70 & 0.02775786 & -0.01203545 & 0.01320180 & -0.00721887 & -0.00665778 \\ 
  0.90 & 0.01743705 & 0.00303553 & 0.01319473 & -0.01785909 & 0.00046374 \\ 
   \hline
\end{tabular}
\caption{Diffeference between the autoregressive coefficients ($\hat{\phi} - \hat{\beta}$) for estimating quantile
$\alpha = $ 0.05. $\phi$ stands for the autoregressive coefficient 
and RSN is the signal to noise ratio. Details for these experiments can 
be found on section \ref{sec:simulation-ar1}} 
\label{tab:sim-auto-005}
\end{table}

% latex table generated in R 3.2.3 by xtable 1.8-2 package
% Tue May 24 17:06:46 2016
\begin{table}[ht]
\centering
\begin{tabular}{rrrrrr}
  \hline
$\phi \backslash RSN$ & 0.01 & 0.05 & 0.1 & 0.5 & 1 \\ 
  \hline
0.25 & -0.00866846 & 0.02731984 & 0.01494079 & -0.01572119 & 0.01000855 \\ 
  0.50 & 0.02547593 & 0.00642516 & 0.00247766 & -0.00871352 & 0.00277974 \\ 
  0.70 & 0.00503748 & 0.00751987 & -0.00107390 & 0.00327506 & -0.00044987 \\ 
  0.90 & 0.00709175 & 0.00692367 & -0.00624776 & 0.00298916 & -0.00493012 \\ 
   \hline
\end{tabular}
\caption{Diffeference between the autoregressive coefficients ($\hat{\phi} - \hat{\beta}$) for estimating quantile
$\alpha = $ 0.1. $\phi$ stands for the autoregressive coefficient 
and RSN is the signal to noise ratio. Details for these experiments can 
be found on section \ref{sec:simulation-ar1}} 
\label{tab:sim-auto-01}
\end{table}

% latex table generated in R 3.2.3 by xtable 1.8-2 package
% Tue May 24 17:06:46 2016
\begin{table}[ht]
\centering
\begin{tabular}{rrrrrr}
  \hline
$\phi \backslash RSN$ & 0.01 & 0.05 & 0.1 & 0.5 & 1 \\ 
  \hline
0.25 & 0.00504450 & -0.00552938 & 0.00198665 & 0.00331038 & 0.00493278 \\ 
  0.50 & 0.00816919 & -0.00518034 & -0.00514326 & 0.00056696 & 0.00347201 \\ 
  0.70 & 0.00817503 & 0.00319246 & -0.00703127 & -0.00478434 & 0.00539972 \\ 
  0.90 & 0.01257091 & 0.00502689 & -0.00469782 & -0.00543626 & 0.00172323 \\ 
   \hline
\end{tabular}
\caption{Diffeference between the autoregressive coefficients ($\hat{\phi} - \hat{\beta}$) for estimating quantile
$\alpha = $ 0.5. $\phi$ stands for the autoregressive coefficient 
and RSN is the signal to noise ratio. Details for these experiments can 
be found on section \ref{sec:simulation-ar1}} 
\label{tab:sim-auto-05}
\end{table}

% latex table generated in R 3.2.3 by xtable 1.8-2 package
% Tue May 24 17:06:46 2016
\begin{table}[ht]
\centering
\begin{tabular}{rrrrrr}
  \hline
$\phi \backslash RSN$ & 0.01 & 0.05 & 0.1 & 0.5 & 1 \\ 
  \hline
0.25 & 0.00063949 & -0.00587333 & -0.00343132 & 0.00057888 & -0.01576153 \\ 
  0.50 & -0.01953741 & -0.00099696 & -0.01220643 & -0.00459181 & -0.02725897 \\ 
  0.70 & 0.00879188 & -0.00578564 & -0.01365016 & -0.01735324 & -0.01595786 \\ 
  0.90 & -0.00432531 & -0.00674863 & 0.00059043 & 0.00040195 & 0.00452383 \\ 
   \hline
\end{tabular}
\caption{Diffeference between the autoregressive coefficients ($\hat{\phi} - \hat{\beta}$) for estimating quantile
$\alpha = $ 0.9. $\phi$ stands for the autoregressive coefficient 
and RSN is the signal to noise ratio. Details for these experiments can 
be found on section \ref{sec:simulation-ar1}} 
\label{tab:sim-auto-09}
\end{table}

% latex table generated in R 3.2.3 by xtable 1.8-2 package
% Tue May 24 17:06:46 2016
\begin{table}[ht]
\centering
\begin{tabular}{rrrrrr}
  \hline
$\phi \backslash RSN$ & 0.01 & 0.05 & 0.1 & 0.5 & 1 \\ 
  \hline
0.25 & -0.00217703 & 0.01773060 & -0.01892853 & 0.01070602 & -0.01105531 \\ 
  0.50 & -0.01530692 & -0.01035399 & -0.02514421 & 0.01278911 & -0.00835432 \\ 
  0.70 & 0.04323376 & -0.00354721 & 0.03589676 & -0.00176410 & -0.00545403 \\ 
  0.90 & 0.04279246 & 0.00727074 & -0.00552405 & 0.00634520 & -0.00118189 \\ 
   \hline
\end{tabular}
\caption{Diffeference between the autoregressive coefficients ($\hat{\phi} - \hat{\beta}$) for estimating quantile
$\alpha = $ 0.95. $\phi$ stands for the autoregressive coefficient 
and RSN is the signal to noise ratio. Details for these experiments can 
be found on section \ref{sec:simulation-ar1}} 
\label{tab:sim-auto-095}
\end{table}


% latex table generated in R 3.2.3 by xtable 1.8-2 package
% Tue May 24 17:06:45 2016
\begin{table}[ht]
\centering
\begin{tabular}{rrrrrr}
  \hline
$\phi \backslash RSN$ & 0.01 & 0.05 & 0.1 & 0.5 & 1 \\ 
  \hline
0.25 & 0.04603811 & -0.01775659 & -0.03214647 & 0.02145871 & -0.00765273 \\ 
  0.50 & 0.03352012 & -0.00448700 & -0.02823629 & -0.04445694 & 0.03234653 \\ 
  0.70 & -0.04953532 & 0.10768253 & -0.03871770 & 0.02006049 & 0.01848257 \\ 
  0.90 & -0.01908312 & 0.05744539 & -0.17970986 & 0.12894696 & 0.02136838 \\ 
   \hline
\end{tabular}
\caption{Coefficient difference between the non-autoregressive part (($\hat{\phi}_0 + z_\alpha  \hat{\sigma}^2_\varepsilon) - \hat{\beta}_0)$ for estimating quantile
$\alpha = $ 0.05. $\phi$ stands for the autoregressive coefficient 
and RSN is the signal to noise ratio. Details for these experiments can 
be found on section \ref{sec:simulation-ar1}} 
\label{tab:sim-intercept-005}
\end{table}

\clearpage
% latex table generated in R 3.2.3 by xtable 1.8-2 package
% Tue May 24 17:06:46 2016
\begin{table}[ht]
\centering
\begin{tabular}{rrrrrr}
  \hline
$\phi \backslash RSN$ & 0.01 & 0.05 & 0.1 & 0.5 & 1 \\ 
  \hline
0.25 & 0.02260184 & -0.04514987 & -0.04187708 & 0.02309700 & -0.02037122 \\ 
  0.50 & -0.03481513 & -0.01041481 & -0.00038617 & -0.00296439 & -0.01172149 \\ 
  0.70 & 0.03459960 & -0.03797767 & 0.01480016 & 0.04972062 & 0.01002299 \\ 
  0.90 & 0.14704364 & -0.06159980 & 0.08091946 & -0.03306211 & 0.01302636 \\ 
   \hline
\end{tabular}
\caption{Coefficient difference between the non-autoregressive part (($\hat{\phi}_0 + z_\alpha  \hat{\sigma}^2_\varepsilon) - \hat{\beta}_0)$ for estimating quantile
$\alpha = $ 0.1. $\phi$ stands for the autoregressive coefficient 
and RSN is the signal to noise ratio. Details for these experiments can 
be found on section \ref{sec:simulation-ar1}} 
\label{tab:sim-intercept-01}
\end{table}

% latex table generated in R 3.2.3 by xtable 1.8-2 package
% Tue May 24 17:06:46 2016
\begin{table}[ht]
\centering
\begin{tabular}{rrrrrr}
  \hline
$\phi \backslash RSN$ & 0.01 & 0.05 & 0.1 & 0.5 & 1 \\ 
  \hline
0.25 & 0.01297018 & -0.00048766 & -0.01041199 & -0.01243439 & 0.02395973 \\ 
  0.50 & -0.00774322 & 0.03340235 & 0.04581712 & -0.00941144 & 0.02834337 \\ 
  0.70 & 0.02631287 & -0.00313672 & 0.05261023 & -0.00929484 & -0.05534836 \\ 
  0.90 & -0.00600514 & -0.01798551 & 0.01424034 & 0.07490435 & -0.10554228 \\ 
   \hline
\end{tabular}
\caption{Coefficient difference between the non-autoregressive part (($\hat{\phi}_0 + z_\alpha  \hat{\sigma}^2_\varepsilon) - \hat{\beta}_0)$ for estimating quantile
$\alpha = $ 0.5. $\phi$ stands for the autoregressive coefficient 
and RSN is the signal to noise ratio. Details for these experiments can 
be found on section \ref{sec:simulation-ar1}} 
\label{tab:sim-intercept-05}
\end{table}
 % clearpage aqui
% latex table generated in R 3.2.3 by xtable 1.8-2 package
% Tue May 24 17:06:46 2016
\begin{table}[ht]
\centering
\begin{tabular}{rrrrrr}
  \hline
$\phi \backslash RSN$ & 0.01 & 0.05 & 0.1 & 0.5 & 1 \\ 
  \hline
0.25 & -0.01239889 & 0.02293524 & -0.02274380 & -0.01803445 & 0.00913629 \\ 
  0.50 & 0.03837217 & 0.01562634 & 0.02285269 & 0.00939266 & 0.08607257 \\ 
  0.70 & 0.05059941 & 0.03251627 & 0.07964660 & 0.06332138 & 0.06101961 \\ 
  0.90 & 0.27272743 & 0.10535168 & 0.04777041 & 0.02227744 & -0.04024329 \\ 
   \hline
\end{tabular}
\caption{Coefficient difference between the non-autoregressive part (($\hat{\phi}_0 + z_\alpha  \hat{\sigma}^2_\varepsilon) - \hat{\beta}_0)$ for estimating quantile
$\alpha = $ 0.9. $\phi$ stands for the autoregressive coefficient 
and RSN is the signal to noise ratio. Details for these experiments can 
be found on section \ref{sec:simulation-ar1}} 
\label{tab:sim-intercept-09}
\end{table}

% latex table generated in R 3.2.3 by xtable 1.8-2 package
% Tue May 24 17:06:46 2016
\begin{table}[ht]
\centering
\begin{tabular}{rrrrrr}
  \hline
$\phi \backslash RSN$ & 0.01 & 0.05 & 0.1 & 0.5 & 1 \\ 
  \hline
0.25 & 0.00335900 & -0.03337376 & 0.03659738 & -0.02854912 & 0.01837794 \\ 
  0.50 & 0.03376431 & -0.00203817 & -0.00638700 & -0.06833761 & -0.00379743 \\ 
  0.70 & -0.07861339 & 0.02610872 & -0.11362453 & 0.05524907 & -0.00006457 \\ 
  0.90 & -0.33397830 & -0.06626543 & 0.05843421 & -0.05809618 & -0.04813865 \\ 
   \hline
\end{tabular}
\caption{Coefficient difference between the non-autoregressive part (($\hat{\phi}_0 + z_\alpha  \hat{\sigma}^2_\varepsilon) - \hat{\beta}_0)$ for estimating quantile
$\alpha = $ 0.95. $\phi$ stands for the autoregressive coefficient 
and RSN is the signal to noise ratio. Details for these experiments can 
be found on section \ref{sec:simulation-ar1}} 
\label{tab:sim-intercept-095}
\end{table}

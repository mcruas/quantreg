\section{The Risk and Ambiguity-Constrained Portfolio Allocation Model}
\label{OptContrStrat}

	In technical literature, there exist many different approaches to characterize the problem of portfolio selection under uncertainty. The classical one is the mean-variance approach introduced by Markowitz in his seminal work \cite{PortSelect_Markovitz}. The idea behind Markowitz's model is to compose a portfolio with the highest expected profit constraining the portfolio's risk to given a fixed level. A key point of this approach is the assumption that an accurate probability description of the uncertain parameters is available. However, in the context of energy trading, this assumption may not be met in practical applications due to the high difficulty on (precisely) characterizing the probability dynamics of the short-term market price. Therefore, decisions are typically made under ambiguity. In this section, we present an extension of the classical risk-constrained approach introduced by Markowitz to incorporate the ETC's aversion to uncertainty on probability modeling.

\subsection{Renewable Production}
\label{SimRenGen}

	The modeling approach adopted in this work to characterize the uncertainty on renewable production follows the ``standard'' scenario-based approach. We argue that physical variables (wind speed and river inflows) generally exhibit a periodical and ``well-behaved'' pattern, especially when simulated on a mid-term aggregated basis (monthly averages, for instance). In this sense, they are suitable for statistical modeling and can be adequately simulated for long-term periods (e.g. more than one year in a monthly basis) without violating their dynamics. The simulation method here used is based on Monte Carlo sampling procedure \cite{Birge_StochProgm} from the periodical stochastic process proposed in \cite{FosteringWPP}. It is important to mention, however, that is out of the scope of this work to discuss adequate statistical models for renewable generation. The scenarios are considered input data and, therefore, exogenous to the model. Nevertheless, we refer to \cite{ReviewProbForecastWP} for an extensive survey on renewable production modeling.

\subsection{Short-term Market Price}
\label{SpotPrice}

	Regardless of market structure, short-term energy market prices are recognized to have a highly volatile pattern and a probability description that is difficult to estimate. The main techniques typically employed to describe their future behavior are based on statistical or fundamentalist approaches. However, as discussed in \cite{RobustSpotPrice}, both methodologies can be easily challenged. For instance, on the one hand, statistical methods heavily relies on the assumption that the historical record available is a good proxy for the future behavior of the market, which can be readily contested especially in markets with high technological development. In addition, the proper choice of the statistical model to describe the stochastic nature of the market prices is a hard task due to their high volatility and non-trivial dependency on variables that are also complex to describe, e.g. GDP, new entrance of market players, etc. On the other hand, fundamentalist approaches are very sensitive to the hypothesis assumed regarding to future dynamics of the market and its structural variables. Therefore, a slight deviation on the initial assumptions may heavily distort the estimated probability distribution of the prices. 

	For these reasons, it is of utmost importance the construction of a portfolio allocation methodology that comprises this modeling uncertainty in order to better characterize the exposure of ETCs to unexpected financial losses. In this work, the hybrid stochastic/robust approach is employed to handle the imprecision on spot price modeling. The idea of the robust counterpart is to act as a ``protection'' against unexpected events and misspecification of the probability nature of the spot prices, \textit{robustifying} the portfolio by means of a worst-case analysis. Methodologically, for each trading period $t \in H$, we assume available a set of spot price \textit{reference} scenarios $(\{\pi^{\text{o}}_{t,\omega}\}_{\omega \in \Omega})$, one for each scenario of renewable production. Such reference scenarios may represent the best spot price description the ETC can provide from available information. Then, a polyhedral set around each reference scenario $\omega \in \Omega$ is defined to delimit the feasible region for worst-case spot price $\boldsymbol{\pi}^{\text{WC}}_{\omega}$. Note that each feasible choice of the spot price within the polyhedral defines a novel distribution, inducing thus a set of probability distributions constructed from the reference one. Therefore, the worst-case analysis within the polyhedral set can be interpreted as a worst-case analysis among a set of probability distributions, characterizing the agent's aversion to ambiguity on the description of the short-term price probability functions. The set of equations (\ref{PUS1})-(\ref{PUS6}) presents the polyhedral set considered in this work.
%
\begin{align}
	& \hspace{-0.40cm} \Pi_{K}(\boldsymbol{\pi}^{\text{o}}_{\omega}) \triangleq \bigg\{ \hspace{-0.40cm} && \boldsymbol{\pi}^{\text{WC}}_{\omega} = \Big[ \pi^{\text{WC}}_{1, \omega}, \dots, \pi^{\text{WC}}_{|H|, \omega} \Big]^{\top} \in \mathbb{R}^{|H|}_{+} ~ \bigg| \label{PUS1} \\
	& && \hspace{-0.60cm} \exists ~ (\boldsymbol{v}_{\omega}^{+}, \boldsymbol{v}_{\omega}^{-}) \in [0, 1]^{|H|} \times [0, 1]^{|H|}; \label {PUS2} \\
	& &&\hspace{-0.60cm} \pi^{\text{WC}}_{t, \omega} = \pi^{\text{o}}_{t, \omega} + \Delta_{t, \omega}^{+} v_{t,\omega}^{+} - \Delta_{t, \omega}^{-} v_{t,\omega}^{-}, ~ \forall ~ t \in H; \label{PUS3} \\
	& &&\hspace{-0.60cm} \sum_{t \in H} (v_{t,\omega}^{+} + v_{t,\omega}^{-}) \leq K; && \label{PUS4} \\
	& &&\hspace{-0.60cm} \pi^{\text{WC}}_{t+1,\omega} \geq (1 - r_{t}^{-}) \pi^{\text{WC}}_{t,\omega}, && \hspace{-2.20cm} \forall ~ t \in \bar{H}; \label{PUS5} \\
	& &&\hspace{-0.60cm} \pi^{\text{WC}}_{t+1,\omega} \leq (1 + r_{t}^{+}) \pi^{\text{WC}}_{t,\omega}, && \hspace{-2.20cm} \forall ~ t \in \bar{H}; ~ \bigg\}. \label{PUS6}
\end{align}

	Equation (\ref{PUS3}) defines the maximum $\Delta_{t, \omega}^{+}$ (\$/MWh) and minimum $\Delta_{t, \omega}^{-}$ (\$/MWh) deviation from the reference scenario $\pi^{\text{o}}_{t, \omega}$ (\$/MWh) the worst-case spot price is allowed to diverge in each period $t \in H$. The percentage magnitude of these deviations are defined by the variables $v_{t,\omega}^{+}$ and $v_{t,\omega}^{-}$. Moreover, expression (\ref{PUS4}) establishes a total budget $(K)$ for the worst-case spot price deviation. Finally, equations (\ref{PUS5})-(\ref{PUS6}) express the relation between consecutive worst-case spot prices, where the set $\bar{H}$ is equal to $H$ except for the ``last'' term, i.e. $\bar{H} = H\setminus\left\{|H|\right\}$. In order to introduce robustness against the imprecision on spot price characterization, the ETC's worst-case revenue is evaluated by the following non-linear optimization model for a given portfolio $\mathbf{x}$, scenario $\omega \in \Omega$ and budget $K$.
%
\begin{align}
	& \hspace{-0.30cm} R_{K}^{\text{WC}}(\mathbf{x}, \boldsymbol{\pi}_{\omega}^{\text{o}}) = \notag \\
			& \hspace{-0.30cm} \min_{\boldsymbol{\pi}_{\omega}^{\text{WC}} \in \Pi_{K}(\boldsymbol{\pi}_{\omega}^{\text{o}})} ~ \sum_{t \in H} \Bigg[ \Bigg(P^{\text{sell}} Q^{\text{sell}} x^{\text{sell}} - \sum_{i \in U} P_{i}^{\text{res}} F_{i}^{\text{res}} x_{i}^{\text{res}}\Bigg) h_{t} \notag \\
	 		& \hspace{1.00cm} + \sum_{j \in C_{t}} \Big( \text{max}\{0, \pi^{\text{WC}}_{t,\omega} - \Gamma_{j}^{\text{call}}\} - P_{j}^{\text{call}} \Big) h_{t} Q_{j}^{\text{call}} x_{j}^{\text{call}} \notag \\
			& \hspace{1.00cm} + \Bigg( \sum_{i \in U} G_{i,t,\omega}^{\text{res}} x_{i}^{\text{res}} - h_{t} Q^{\text{sell}} x^{\text{sell}} \Bigg)\pi^{\text{WC}}_{t,\omega} \Bigg]; \label{RobustBusinessRevenue}
\end{align}

	The ETC's worst-case revenue (\ref{RobustBusinessRevenue}) is constructed to identify the sequence of spot prices within the polyhedral uncertainty set (\ref{PUS1})-(\ref{PUS6}) that results in the worst possible trading cash-flow. It is important to notice that the budget parameter $K$ defines a $K$-neighborhood of feasible spot price values. Therefore, as we increase the value of $K$, the set $\Pi_{K}(\boldsymbol{\pi}_{\omega}^{\text{o}})$ expands around the reference scenario, allowing the optimization model to produce more adversity to the ETC's cash flow. As a consequence, the higher is the value of $K$, the more conservative is the worst-case solution. Finally, note that the naive solution that establishes extreme values for the worst-case spot price (cap/floor values) may not be optimal due to the presence of call options in the portfolio. For instance, for high values of spot price, the call option payoff is also high leading to a high worst-case revenue.

\subsection{Portfolio Selection Model}
\label{PortfAllocaModel}

	The main objective of this work is to provide a risk and ambiguity-constrained portfolio selection model of electricity contracts. In order to construct the risk and ambiguity constraints, we resort to the widely used $\alpha$-percentile risk measure, the $\alpha$--Conditional Value-at-Risk ($\text{CVaR}_{\alpha}$) \cite{OptimizCVaR}. Structurally, the proposed model is composed by the classical expected value maximization framework with a $\text{CVaR}_{\alpha}$ constraint to a minimal profit requirement $\underline{R}^{\text{risk}}$ (\$) \cite{RiskConstPortSelect}. In addition, in order to handle the imprecision on spot price characterization, a set of $\mathcal{K}$ ambiguity constraints are introduced in the model to restrict the worst-case revenue (\ref{RobustBusinessRevenue}) to minimum levels $\underline{R}_{K}^{\text{amb}}$ (\$). Therefore, we are not only expressing the risk-aversion preferences over stochastic cash-flows by means of a combination of expected value maximization with risk constraint but also managing the imprecision on probability modeling via $\text{CVaR}_{\alpha}$ constraints over the worst-case revenue. The mathematical formulation of the proposed model is:
%
\begin{align}
	& \hspace{-0.06cm} \Maximize_{\substack{\mathbf{x}}} ~ \mathbb{E}\Big[R(\mathbf{x}, \accentset{\sim}{\boldsymbol{\pi}})\Big] \label{FullFirstLevel1} \\
	& \text{subject to:} \notag \\
	& \text{CVaR}_{\alpha}\Big(R(\mathbf{x}, \accentset{\sim}{\boldsymbol{\pi}})\Big) \geq \underline{R}^{\text{risk}}; \label{FullFirstLevel2} \\
	& \text{CVaR}_{\alpha}\Big(R_{K}^{\text{WC}}(\mathbf{x}, \accentset{\sim}{\boldsymbol{\pi}}^{\text{o}})\Big) \geq \underline{R}_{K}^{\text{amb}}, & \hspace{-1.00cm} \forall ~ K \in \mathcal{K}; \label{FullFirstLevel3} \\
	& Q^{\text{sell}} x^{\text{sell}} \leq \sum_{i \in U} F_{i}^{\text{res}} x^{\text{res}}_{i}; & \label{FullFirstLevel4} \\
	& x^{\text{sell}}, x_{i}^{\text{res}}, x_{j}^{\text{call}} \in [0,1], & \hspace{-1.00cm} \forall ~ i \in U, j \in C_{t}, t \in H, \label{FullFirstLevel5}
\end{align}
%
where equations (\ref{FullFirstLevel1}) and (\ref{FullFirstLevel2}) represents the classical risk-constrained expected value maximization modeling; (\ref{FullFirstLevel3}) defines the set of ambiguity constraints, one for each budget level $K \in \mathcal{K}$. For nomenclature purposes, we will call $\accentset{\sim}{\boldsymbol{\pi}}$ as \textit{nominal} random spot price; (\ref{FullFirstLevel4}) ensures that the amount sold on the supply contract is fully backed by the renewable portfolio; and (\ref{FullFirstLevel5}) defines the decision variable bounds. Since we are assuming a discrete uncertainty environment (discrete sample space), the $\text{CVaR}_{\alpha}$ calculation on equations (\ref{FullFirstLevel2}) and (\ref{FullFirstLevel3}) can be written with linear constraints \cite{OptimizCVaR}. Therefore, for a given worst-case revenue, the optimization problem (\ref{FullFirstLevel2})-(\ref{FullFirstLevel5}) has an equivalent linear formulation suitable for off-the-shelf solvers such as \cite{Xpress}. In section \ref{SolMeth}, a methodology to handle the difficulty on the worst-case revenue assessment is discussed.

	It is important to highlight that, in order to present a general formulation, the nominal spot price random variable in (\ref{FullFirstLevel1})-(\ref{FullFirstLevel3}) may be (potentially) different from the one that acts as reference in the polyhedral set (\ref{PUS1})-(\ref{PUS6}), used to compute the worst-case revenue (\ref{FullFirstLevel3}). Naturally, a possible formulation is to make $\accentset{\sim}{\boldsymbol{\pi}} = \accentset{\sim}{\boldsymbol{\pi}}^{\text{o}}$. In this context, the ambiguity constraints can be interpreted as a set of restrictions to account for the misspecification of the spot price characterization used to compute the risk and return in the Markowitz framework. We highlight that such framework is the typical approach used in ambiguity theory to model uncertainty on probability modeling, i.e. a set of probability distributions constructed from an \textit{a priori} one. However, we argue that different specifications for $\accentset{\sim}{\boldsymbol{\pi}}$ and $\accentset{\sim}{\boldsymbol{\pi}}^{\text{o}}$ are also rational. For instance, following the ideas discussed in \cite{RobustSpotPrice}, the ETC may want to optimize the portfolio considering the risk and return evaluated by $\accentset{\sim}{\boldsymbol{\pi}}$ (e.g. the best spot price characterization the agent can provide from available information), but restricts the revenue of the portfolio to a set of endogenously-defined \textit{stress} analysis. In this framework, the reference scenarios are fixed to a sequence of stress ``scenarios'' and a portfolio-adjusted \textit{stress revenue} is constrained to each minimum level $\underline{R}_{K}^{\text{amb}}$. Finally, the minimal profit requirements necessary to construct the set of ambiguity constraints (\ref{FullFirstLevel3}) can be extracted from corporative financial parameters, i.e. financial values that change the ETC's financial \textit{status quo}. For instance, if the trading company cannot admit a financial loss, then the agent can specify $\underline{R}_{K}^{\text{amb}} = 0$ for some high value of $K$ (e.g., $K =3$).
\section{Estimating distribution function from quantile regressions}
\label{sec:estimating-distribution}

In many applications where a time series model is employed, we often consider the innovations' distribution as known. Take, for example, the AR(p) model:
$$y_{t}=c+\varepsilon_{t}+\sum_{i=1}^{p}\phi_{i}y_{t-1}$$
In this model, errors $\varepsilon_t$ are assumed to have normal distribution with zero mean. 

When we are dealing with meteorological time series, however, we can't always assume normality. In these cases, one can either find a distribution that has a better fit to the data or have a nonparametric method to estimate the distribution directly from the available data.

In a time series framework, where a time series $y_t$ is given by a linear model of its regressors $x_t$
$$y_t = \beta^T x_t + \varepsilon_t,$$
we propose to estimate the $k$-step ahead distribution of $y_t$ with a nonparametric approach.
Let an empirical $\alpha$-quantile $\hat{q}_\alpha \in \mathcal{Q}$ be a functional belonging to a functional space. In any given $t$, we can estimate the sequence of quantiles $\{ q_{\alpha}(x_t) \}_{\alpha \in A}$ by solving the problem defined on equations (\ref{eq:non-crossing-quantiles1})-(\ref{eq:non-crossing-constraint}). 
 After evaluating this sequence, by making equal 
\begin{equation}
\hat{Q}_{y_t|X}(\alpha) = \hat{q}_\alpha(x_t), \qquad \forall \alpha \in A,
\end{equation}
we have a set of size $|A|$ of values to define the discrete function over the first argument $\hat{Q}_{y_t|x_t}(\alpha,X=x_t): A \times \mathbb{R}^d \rightarrow \mathbb{R}$. The goal of having function $\hat{Q}$ is to use it as base to construct the estimated quantile function $\hat{Q}'_{y_t|X=x_t}(\alpha,x_t): [0,1] \times \mathbb{R}^d \rightarrow \mathbb{R}$. 

A problem arises for the distribution extremities, because when $\alpha = 0$ or $\alpha = 1$, the optimization problem becomes unbounded. In order to find values for $\hat{Q}(\alpha,x_t)$ when $\alpha \in \{0,1\}$, we chose to linearly extrapolate its values. %completar com explicação de extrapolação
Note that as $A \subset [0,1]$, the domain of $\hat{Q}$ is also a subset of the domain of $\hat{Q}'$. 
The estimative of $\hat{Q}'$ is done by interpolating points of $\hat{Q}$ over the interval $[0,1]$.
Thus, the distribution found for $\hat{y}_{\tau}$ is nonparametric, as no previous assumptions are made about its shape, and its form is fully recovered by the data we have.


We investigate two different approaches for $Q_{y_t}$ by the functional structure of each individual $q_\alpha(x_t)$.
In section \ref{sec:linear-models}, we explore the case where the individual quantiles $q_\alpha(x_t)$ are a linear function of its arguments:
\begin{equation}
\hat{q}_\alpha(x_t) = \beta_{0,\alpha} +   \beta_\alpha^T x_t,
\label{eq:fun-quantile}
\end{equation}
where $\beta^\alpha$ is a vector of coefficients for the explanatory variables.

In section \ref{sec:npqar} we introduce a Nonparametric Quantile Autoregressive model with a $\ell_{1}$-penalty term, in order to properly simulate densities for several $\alpha$-quantiles. In this nonparametric approach we don't assume any form for $q_\alpha(x_t)$, but rather let the function adjust to the data. To prevent overfitting, the $\ell_1$ penalty for the second derivative (approximated by the second difference of the ordered observations) is included in the objective function. The result of this optimization problem is that each $q_\alpha(x_t)$ will be a function with finite second derivative.

In order to find good estimates for $Q_{y_{t}}(\alpha)$ when $\alpha$ approaches 0 or 1, as well as performing interpolation on the values that were not directly estimated, we can either use a kernel smoothing function, splines, linear approximation, or any other method. 
\textbf{This will be developed later.}


\section{Linear Models for the Quantile Autoregression}
\label{sec:linear-models}

Given a time series $\{y_t\}$, we investigate how to select which lags will be included in the Quantile Autoregression. We won't be choosing the full model because this normally leads to a bigger variance in our estimators, which is often linked with bad performance in forecasting applications. So our strategy will be to use some sort of regularization method in order to improve performance.
We investigate two ways of accomplishing this goal.
The first of them consists of selecting the best subset of variables through Mixed Integer Programming, given that $K$ variables are included in the model. Using MIP to select the best subset of variables is investigated in \cite{bertsimas2015best}. The second way is including a $\ell_1$ penalty on the linear quantile regression, as in \cite{kim2009ell_1}, and let the model select which and how many variables will have nonzero coefficients. 
Both of them will be built over the standard Quantile Linear Regression model. In the end of the section, we discuss a information criteria to be used for quantile regression and verify how close are the solutions in the eyes of this criteria.

When we choose $q(x_t)$ to be a linear function, as on equation \ref{eq:linear-model} (that we reproduce below for convenience):
\begin{equation}
\min_{f}\sum_{t=1}^{n}\alpha|y_{t}-q(x_t)|^{+}+(1-\alpha)|y_{t}-q(x_t)|^{-},
\end{equation}
we can substitute it on problem \ref{eq:qar-general}, getting the following LP problem:
\begin{equation}
\begin{aligned}\min_{\beta_0,\beta,\varepsilon_{t}^{+},\varepsilon_{t}^{-}} & \sum_{t=1}^{n}\left(\alpha\varepsilon_{t}^{+}+(1-\alpha)\varepsilon_{t}^{-}\right)\\
\mbox{s.t. } & \varepsilon_{t}^{+}-\varepsilon_{t}^{-}=y_{t} - \beta_0 - \beta^T x_{t},\qquad\forall t\in\{1,\dots,n\},\\
& \varepsilon_t^+,\varepsilon_t^- \geq 0, \qquad \forall t \in \{1,\dots,n\}.
\end{aligned}
\label{eq:qar-lp}
\end{equation}

In this work, we didn't explore the addition of terms other than the terms $y_t$ past lags. For example, we could include functions of $y_{t-p}$, such as $log(y_{t-p})$ or $exp(y_{t-p})$. We leave such inclusion for further works. 

\subsection{Best subset selection with Mixed Integer Programming}
\label{sec:best-subset-mip}

In this part, we investigate the usage of Mixed Integer Programming to select which variables are included in the model, up to a limit of inclusions imposed \textit{a priori}. The optimization problem is described below:
\begin{eqnarray}
\min_{\beta_{0},\beta, z,\varepsilon_{t}^{+},\varepsilon_{t}^{-}} &  \sum_{t=1}^{n}\left(\alpha\varepsilon_{t}^{+}+(1-\alpha)\varepsilon_{t}^{-}\right) \\
\mbox{s.t } & \varepsilon_{t}^{+}-\varepsilon_{t}^{-}=y_{t}-\beta_{0}-\sum_{p=1}^{P}\beta_{p}x_{t,p},& \qquad\forall t\in\{1,\dots,n\}, \label{linear1}\\
 & \varepsilon_{t}^{+},\varepsilon_{t}^{-}\geq0,&\qquad\forall t \in \{1,\dots,n\}, \label{linear2}\\
 & - M_U z_p \leq \beta_p \leq M_U z_p,&\qquad\forall p\in\{1,\dots,P\}, \label{linear3}\\
 & \sum_{p=1}^P z_p \leq K, \label{linear4}\\
 & z_p \in \{0,1\},&\qquad\forall p\in\{1,\dots,P\}. \label{eq:linear5}
\end{eqnarray}
The objective function and constraints (\ref{linear2}) and (\ref{linear3}) are those from the standard linear quantile regression. The other constraints implement the process of regularization, forcing a maximum of $K$ variables to be included. By (\ref{linear3}), variable $z_p$ is a binary that assumes 1 when the coefficient $\beta_p$ is included. $M_U$ is chosen in order to guarantee that $M_U \geq \|\hat{\beta}\|_{\infty}$. The solution given by $\beta_0$ and $\beta$ will be the best linear quantile regression with $K$ nonzero coefficients. 

We ran this optimization for each value of $K \in \{1, \dots, 12\}$ and quantiles $\alpha \in \{0.05, 0.1, 0.5, 0.9, 0.95\}$. We could see that for all quantiles the 12\textsuperscript{th} lag was the one included when $K=1$. When $K=2$, the 1\textsuperscript{st} lag was always included, sometimes with $\beta_{12}$, some others with $\beta_4$ and once with $\beta_{11}$. These 4 lags that were present until now are the only ones selected when $K=3$. For $K=4$, those same four lags were selected for three quantiles (0.05, 0.1 and 0.5), but for the others (0.9 and 0.95) we have $\beta_6$, $\beta_7$ and $\beta_9$ also as selected. From now on, the inclusion of more lags represent a lower increase in the fit of the quantile regression. The estimated coefficient values for all $K$'s are available in the appendices section. Figure \ref{fig:icaraizinho-crossing-200} shows a linear estimator for the quantiles $(0.05, 0.1, 0.25, 0.5, 0.75, 0.9, 0.95)$.

\begin{figure}
\centering
\includegraphics[width=0.7\linewidth]{Figuras/npqar/icaraizinho-crossing-200}
\caption{Linear Quantile Regression when only $y_{t-1}$ is used as explanatory variable}
\label{fig:icaraizinho-crossing-200}
\end{figure}


\subsection{Best subset selection with a $\ell_1$ penalty}
\label{sec:best-subset-ell1}

Another way of doing regularization is including the $\ell_1$-norm of the coefficients on the objective function. The advantage of this method is that coefficients are shrunk towards zero, and only some of them will have nonzero coefficients. By lowering the penalty we impose on the $\ell_1$-norm, more variables are being added to the model. 
This is the same strategy of the LASSO, and its usage for the quantile regression is discussed in \cite{li2012l1}.
The proposed optimization problem to be solved is:

\begin{equation}
\min_{\beta_{0},\beta}\sum_{t=1}^{n}\alpha|y_{t}-q(x_t)|^{+}+(1-\alpha)|y_{t}-q(x_t)|^{-}+\lambda\|\beta\|_{1}
\label{eq:l1-qar-optim}
\end{equation}
\[
q(x_t)=\beta_{0}-\sum_{p=1}^{P}\beta_{p}x_{t,p},
\]
where the regressors $x_{t,p}$ used are its lags. In order to represent the above problem to be solved with linear programming solver, we restructure the problem as below:
\begin{eqnarray}
\beta_\lambda^{*LASSO} = \argmin_{\beta_{0},\beta,\varepsilon_{t}^{+},\varepsilon_{t}^{-}} & \sum_{i=1}^{n}\left(\alpha\varepsilon_{t}^{+}+(1-\alpha)\varepsilon_{t}^{-}\right)+\lambda\sum_{p=1}^{P}\mbox{\ensuremath{\xi}}_{p} \label{eq:obj-lasso} \\
\mbox{s.t. } & \varepsilon_{t}^{+}-\varepsilon_{t}^{-}=y_{t}-\beta_{0}-\sum_{p=1}^{P}\beta_{p}x_{t,p},\qquad\forall t\in\{1,\dots,n\},\\
 & \varepsilon_{t}^{+},\varepsilon_{t}^{-}\geq0,\qquad\forall t \in \{1,\dots,n\},\\
 & \xi_{p}\geq\beta_{p},\qquad\forall p\in\{1,\dots,P\}, \label{l1-qar-3}\\
 & \xi_{p}\geq-\beta_{p},\qquad\forall p\in\{1,\dots,P\}, \label{l1-qar-4}
\end{eqnarray}
Once again, this model is built upon the standard linear programming model for the quantile regression (equation \ref{eq:qar-lp}). 
On the above formulation, the $\ell_1$ norm of equation (\ref{eq:l1-qar-optim}) is substituted by the sum of $\xi_p$, which represents the absolute value of $\beta_p$. The link between variables $\xi_p$ and $\beta_p$ is made by constraints (\ref{l1-qar-3}) and (\ref{l1-qar-4}). Note that the linear coefficient $\beta_0$ is not included in the penalization, as the sum of penalties on the objective function \ref{eq:obj-lasso}.

For such estimation to produce good results, however, each variable must have the same relative weight in comparison with one another. So, before solving the optimization problem, we normalize all variables to have mean $\mu = 0$ and variance $\sigma^2 = 1$. For the vector of observations for each covariate (that in our problem represents is a vector of observations of lags $y_{t-p}$), we apply the transformation $\tilde{y}_{t-p,i} = (y_{t-p,i} - \bar{y}_{t-p}) / \sigma_{t-p}$, where $\bar{y}_{t-p}$ is the $p$-lag mean and $\sigma_{t-p}$ the $p$-lag standard deviation. We use the $\tilde{y}_{t-p,i}$ series to estimate the coefficients. Once done that, we multiply each coefficient for its standard deviation to get the correct coefficient: $\beta_i = \tilde{\beta}_i \dot \sigma_{t-p}$.

For low values of $\lambda$, the penalty is small and thus we have a model where all coefficients have a nonzero value. On the other hand, while $\lambda$ is increased the coefficients shrink towards zero; in the limit we have a constant model. For instance, we don't penalize the linear coefficient $\beta_0$. For the same quantiles values $\alpha$ we experimented on section \ref{sec:best-subset-mip} ($\alpha \in \{0.05, 0.1, 0.5, 0.9, 0.95\}$). 


\begin{figure}
  \centering
  \begin{minipage}[t]{0.4\linewidth}
    \centering
    \begin{minipage}[t]{\linewidth}
      \centering     \includegraphics[width=\textwidth]{Figuras/selecao-lasso/par-sellasso-005.pdf}
    \end{minipage}
    \begin{minipage}[b]{\linewidth}
      \centering     \includegraphics[width=\textwidth]{Figuras/selecao-lasso/par-sellasso-01.pdf}
    \end{minipage}
     \begin{minipage}[b]{\linewidth}
      \centering     \includegraphics[width=\textwidth]{Figuras/selecao-lasso/par-sellasso-05.pdf}
     \end{minipage}
  \end{minipage}
  \begin{minipage}[t]{0.4\linewidth}
    \centering
    \begin{minipage}[b]{\linewidth}
      \centering     \includegraphics[width=\textwidth]{Figuras/selecao-lasso/par-sellasso-09.pdf}
    \end{minipage}
     \begin{minipage}[b]{\linewidth}
      \centering     \includegraphics[width=\textwidth]{Figuras/selecao-lasso/par-sellasso-095.pdf}
      \label{fig:npqar-cross}
     \end{minipage}
  \end{minipage}
  \caption{Coefficients path for a few different values of $\alpha$-quantiles. $\lambda$ is presented in a $\log_{10}$ scale, to make visualization easier.}
  \label{fig:npqar-results}
\end{figure}

It is important to mention that even though we have coefficients that are estimated by this method, we don't use them directly. Instead, the nonzero coefficients will be the only covariates used as explanatory variables of a regular quantile autoregression, solved by the linear programming problem \ref{eq:qar-lp}. In summary, the optimization in equation \ref{eq:l1-qar-optim} acts as a variable selection for the subsequent estimation, which is normally called the post-lasso estimation \cite{belloni2009least}.

We are interested, finally, in finding the post-lasso coefficients $\beta_\lambda^*$, which is the solution of the optimization problem given below:
\begin{equation}
\begin{aligned} \beta_\lambda^{*} = \argmin_{\beta_0,\beta,\varepsilon_{t}^{+},\varepsilon_{t}^{-}} & \sum_{t=1}^{n}\left(\alpha\varepsilon_{t}^{+}+(1-\alpha)\varepsilon_{t}^{-}\right)\\
\mbox{s.t. } & \varepsilon_{t}^{+}-\varepsilon_{t}^{-}=y_{t} - \beta_0 - \sum_{p\in L_\lambda} \beta_p x_{t,p},\qquad\forall t\in\{1,\dots,n\},\\
& \varepsilon_t^+,\varepsilon_t^- \geq 0, \qquad \forall t \in \{1,\dots,n\}.
\end{aligned}
\label{eq:post-lasso}
\end{equation}
Note that only a subset of the $P$ covariates will have nonzero values, which are given by the set 
\begin{equation*}
L_\lambda = \{ p \; | \; p \in \{ 1,\dots,P \}, \; |\beta^{*LASSO}_{\lambda,p}| \neq 0  \}.
\end{equation*}
Hence, we have that
$$\beta^{*LASSO}_{\lambda,p} = 0 \iff \beta^{*}_{\lambda,p} = 0.$$


\subsection{Simulation Study}
\label{sec:simulation-ar1}
If we knew an autoregressive process true model
\[
	y_t = \phi_0 + \sum_{p=1}^{P} \phi_p y_{t-p} + \varepsilon_t,
\]
and knew the error distribution, we could estimate the one-step ahead quantile as the sum $\hat{y}_{t+1} + t_{\alpha}$, where $t_\alpha$ is the $\alpha$-quantile for the error distribution. This means that we would be able to use all developments made on conditional mean estimation and simply add an error to estimate quantiles. 

However, that is not the case. When working with quantile estimation for real data we don't know the generating process exactly. Using a quantile regression model provides us a good solution even without making assumptions about the error distribution. 

We propose simulating an AR(1) model
\begin{equation}
	y_t = \phi_0 +  \phi y_{t-1} + \varepsilon_t,\qquad \varepsilon_t \sim N(0, \sigma_\varepsilon^2),	
\label{eq:sim-true-model}
\end{equation}
and test two approaches to predict the one-step ahead quantile. On the first one, we consider known the process true model, given on equation \ref{eq:sim-true-model}. Thus, our task is to estimate values for $\hat{\phi}_0$, $\hat{\phi}$ and $\hat{\sigma}_\epsilon^2$. In order to calculate the one-step ahead $\alpha$-quantile, we need to compute
\begin{equation}
	\hat{q}^{AR}_{t+1|t}(\alpha) = \hat{y}_{t+1|t} + z_\alpha \hat{\sigma}_\varepsilon,
\end{equation}
where $\hat{y}_{t+1|t} = \hat{\phi}_0 + \hat{\phi} y_{t}$ stands for the one-step ahead conditional mean and $z_\alpha = F^{-1}(\alpha)$, where $F$ is the gaussian distribution function.

On the second approach, we fit a quantile regression by solving problem \ref{eq:qar-lp}. The solution of this optimization problem are coefficients $\hat{\beta}_0$ and $\hat{\beta}$. In order to find the one-step ahead $\alpha$-quantile, we use the following expression:
\begin{equation}
		\hat{q}^{QR}_{t+1|t}(\alpha) = \hat{\beta}_0 + \hat{\beta} y_{t}.
\end{equation}
Note that in both approaches we have one intercept term ($z_\alpha \sigma_\varepsilon + {\phi}_0$ and $\beta_0$) and a coefficient for the first lag ($\phi$ and $\beta$). 

We generate data according to equation \ref{eq:sim-true-model}, with different values for $\phi$ (0.25, 0.5, 0.7 and 0.9) and different signal to noise ratios (0.01, 0.05, 0.1, 0.5, 1). We use the signal to noise ratio (RSN) to form the error variance such that $\sigma_e^2 = \dfrac{\phi_0}{1-\phi} \cdot RSN$. This experiment was run with samples of size $n=20000$. The first half was used to fit coefficients and the second half was used as testing set, on which forecasting was done.

Our first goal is then to evaluate the ability of these two approaches to predict the one-step ahead $\alpha$-quantile for a few selected $\alpha$'s. We define the model $m$ forecasting error $\varepsilon_t^m$ as the quantity 
\begin{equation}
\varepsilon_t^m(\alpha) = \hat{q}^m_{t|t-1}(\alpha) - q_t(\alpha) = \hat{q}^m_{t|t-1}(\alpha) - \phi_0 - \phi y_{t-1} - z_\alpha \sigma_\epsilon.
\end{equation}
We use the Root Mean Squared Error (RMSE) to evaluate forecasting performance, which is defined for the model $m$ as follows:
\begin{equation}
RMSE^m = \sqrt{ \frac{1}{n-1} \sum_{i=2}^{n} \left( \varepsilon_t^m(\alpha) \right)^2}
\end{equation}

On section \ref{sec:simulation-tables}, we show a comparison of RMSE for both approaches. To compare them, we compute their RMSE ratio
\begin{equation}
R^{Q/A} = \dfrac{RMSE^{QR}}{RMSE^{AR}}.
\end{equation}
If $R^{Q/A}$ is smaller than 1, it means that the quantile regression approach performed better in terms of RMSE than the conditional mean based on autoregressive models approach.

Once simulations are executed, we notice forecasting errors are similar for both approaches. The exception being when $\phi$ is large (0.9) and the noise is small ($RSN = 0.01$). In those cases, quantile regression performed on average 10\% better than when using the conditional mean approach.
We also provide tables showing differences for the estimated autoregressive and intercept coefficients on section \ref{sec:simulation-tables}.

\subsection{Model selection}

On sections \ref{sec:best-subset-mip} and \ref{sec:best-subset-ell1}, we presented ways of doing regularization. But regularization can be done with different levels of parsimony. For example, we can select a different number $K$ of variables to be included in the best subset selection via MIP or choose different values of $\lambda$ for the $\ell_1$ penalty. Each of these choices leads to a different model, so one needs to know how to select the best one among the options we have. One way of achieving this is by using an information criteria to guide our decision. 

An information criteria summarizes two aspects. One of them refers to how well the model fits the in-sample observations. The other part penalizes the quantity of covariates used in the model. By penalizing how big our model is, we prevent overfitting from happening. So, in order to enter the model, the covariate must supply enough goodness of fit.
In \cite{machado1993robust}, it is presented a variation of the Schwarz criteria for M-estimators that includes quantile regression. The Schwarz Information Criteria (SIC) adapted to the quantile autoregression case is presented below:
\begin{align} 
\begin{split}
SIC(j) = n \log(\hat{\sigma}_{j})+\frac{1}{2}p_{j}\log n,\label{eq:SIC}
\end{split}					
\end{align}
where $\hat{\sigma}_{j}= \frac{1}{n} \sum_{t=1}^{n} \rho_{\alpha}(y_{t}-x^{T}_{t}\hat{\beta}_{n}(\alpha))$, $\rho_\alpha(\cdot)$  is the penalization function and $p_{j}$ the $j$\textsuperscript{th} model's dimension. This procedure leads to a consistent model selection if the model is well specified. 


% % % Métrica

Optimizing a LP problem is many times faster than a similar-sized MIP problem. One of our goals is to test whether a solution of a model with a $\ell_1$-norm can approximate well a solution given by the MIP problem. We propose an experiment that is described as follows. First, we calculate the quantity $k(\lambda)$ of nonzero coefficients, for each given lambda:
\begin{equation}
k_\lambda = \| \beta^*_\lambda \|_0.
\end{equation}
Then, for each number $K$ of total nonzero coefficients (from 1 until 13, where 1 means that only the intercept is included), there will be a penalty $\lambda^*_K$ which minimizes the SIC:
\begin{equation}
\lambda^*_K = \argmin_\lambda \left\lbrace  SIC\left( k_\lambda \right)  | \, k_\lambda = K \right\rbrace.
\end{equation}
Thus, we can compare the SIC of the best lasso fit where exactly $K$ variables are selected with the SIC selected by the MIP problem, also with $K$ variables selected.

To help us view the difference of results between both methods, we define a distance metric $d$ between the subset of coefficients chosen by each one of them. Let 
\begin{equation}
d(\beta^*_{MIP(K)}, \beta^*_{\lambda^*_K}) =  \frac{1}{2K} \sum_{p=1}^P { \left| I(\beta^*_{MIP(K),p}) - I( \beta^*_{\lambda^*_K,p}) \right| }, 
\label{eq:distance}
\end{equation}
where $I$ is an indicator function such that $I(x) = 0$ if $x = 0$ and $I(x)=1$ otherwise. 

Figure \ref{fig:comparison-lm-results} shows the results of these experiments for quantiles $\alpha \in \{0.05, 0.1, 0.5, 0.9, 0.95\}$. The results point us that for small values of $K$ the distance between coefficients is bigger and where we observe the biggest differences between the SIC values. The minimum SIC value for the MIP problem is usually found between 4 and 6 variables in the model.


\begin{figure}
  \centering
  \begin{minipage}[t]{0.4\linewidth}
    \centering
    \begin{minipage}[t]{\linewidth}
      \centering     \includegraphics[width=\textwidth]{Figuras/SIC005.pdf}
    \end{minipage}
    \begin{minipage}[b]{\linewidth}
      \centering     \includegraphics[width=\textwidth]{Figuras/SIC01.pdf}
    \end{minipage}
     \begin{minipage}[b]{\linewidth}
      \centering     \includegraphics[width=\textwidth]{Figuras/SIC05.pdf}
     \end{minipage}
  \end{minipage}
  \begin{minipage}[t]{0.4\linewidth}
    \centering
    \begin{minipage}[b]{\linewidth}
      \centering     \includegraphics[width=\textwidth]{Figuras/SIC09.pdf}
    \end{minipage}
     \begin{minipage}[b]{\linewidth}
      \centering     \includegraphics[width=\textwidth]{Figuras/SIC095.pdf}
      \label{fig:npqar-cross}
     \end{minipage}
  \end{minipage}
  \caption{Comparison of SIC between a solution with Lasso as a variable selector and the best subset selection with MIP. The bars represent the distance $d$ as defined by equation \ref{eq:distance}. \\ (*) When the distance is zero, it means that the same variables are selected from both methods for a given $k$. Thus, in these cases we have the same SIC for both of them.}
  \label{fig:comparison-lm-results}
\end{figure}




\subsection{Quantile Autoregression with a nonparametric approach}
\label{sec:npqar}

Fitting a linear estimator for the Quantile Auto Regression isn't appropriate  when nonlinearity is present in the data. This nonlinearity may produce a linear estimator that underestimates the quantile for a chunk of data while overestimating for the other chunk. To prevent this issue from occurring we propose a modification which we let the prediction $q_\alpha(x_t)$ adjust freely to the data and its nonlinearities. To prevent overfitting and smoothen our predictor, we include a penalty on its roughness by including the $\ell_1$ norm of its second derivative. For more information on the $\ell_1$ norm acting as a filter, one can refer to \cite{kim2009ell_1}.

This time, as opposed to when employing linear models, we don't suppose any functional form for $q_\alpha(x_t)$. This forces us to build each $q_\alpha$ differently: instead of finding a set of parameters that fully defines the function, we find a value for $q_\alpha(x_t)$ at each instant $t$. On the optimization problem, we will find optimal values for a variable $q_{\alpha t} \in \mathbb{R}$, each consisting of a single point. The sequence of $\{ q^*_{\alpha t} \} $ will provide a discretization for the full function $\hat{q}_\alpha(x_t)$, which can be found by interpolating these points.

% notação estatística de ordem. com x^(0)

Let $\{\tilde{y}_t \}_{t=1}^n$ be the sequence of observations in time $t$. Now, let $\tilde{x}_t$ be the $p-$lagged time series of $\tilde{y}_t$, such that $\tilde{x}_t = L^p(\tilde{y}_t)$, where $L$ is the lag operator. Matching each observation $\tilde{y}_t$ with its $p-$lagged correspondent $\tilde{x}_t$ will produce $n-p$ pairs $\{(\tilde{y}_t,\tilde{x}_t)\}_{t=p+1}^n$ (note that the first $p$ observations of $y_t$ must be discarded). When we order the observation of $x$ in such way that they are in growing order
$$\tilde{x}^{(p+1)} \leq \tilde{x}^{(p+2)} \leq \dots \leq \tilde{x}^{(n)},$$ 
we can then define $\{x_i\}_{i=1}^{n-p} = \{\tilde{x}^{(t)} \}_{t=p+1}^{n}$ and $\{y_i\}_{i=1}^{n-p} = \{\tilde{y}^{(t)} \}_{t=p+1}^{n}$ and $T' = \{2,\dots, n-p-1\}$. As we need the second difference of $q_i$, $I$ has to be shortened by two elements.

Our optimization model to estimate the nonparametric quantile is as follows:
\begin{equation}
\begin{split}
\hat{q}_\alpha(x_t) =\underset{q_{\alpha t}}{\arg\min}\sum_{t\in T'} \left( \alpha |y_{t}-q_{\alpha t}|^{+} + (1-\alpha)|y_{t}-q_{\alpha t}|^{-}\right) \\ +\lambda_1  \sum_{t\in T'}|D_{x_t}^{1}q_{\alpha t}| +\lambda_2  \sum_{t\in T'}|D_{x_t}^{2}q_{\alpha t}|,
\end{split}
\end{equation}
where $D^1 q_t$ and $D^2 q_t$ are the first and second derivatives of the $q_\alpha(x_t)$ function, calculated as follows:
\begin{equation*}
D_{x_{t}}^{2}q_{\alpha t}=\frac{\left(\frac{q_{\alpha t+1}-q_{\alpha t}}{x_{t+1}-x_{t}}\right)-\left(\frac{q_{\alpha t}-q_{\alpha t-1}}{x_{t}-x_{t-1}}\right)}{x_{t+1}-2x_{t} + x_{t-1}},
\end{equation*}


\begin{equation*}
D^{1}_{t \alpha }=\frac{q_{\alpha t+1}-q_{\alpha t}}{x_{t+1}-x_{t}}.
\end{equation*}
The first part on the objective function is the usual quantile regression condition for $\{q_{t\alpha}\}_{\alpha \in A}$. The second part is the $\ell_1$-filter. The purpose of a filter is to control the amount of variation for our estimator $q_\alpha(x_t)$. When no penalty is employed we would always get $q_{\alpha t} = y_t$, for any given $\alpha$. On the other hand, when $\lambda \rightarrow \infty$, our estimator approaches the linear quantile regression. 

The full model can be rewritten as a LP problem as bellow:
\begin{eqnarray}
\min_{q_{\alpha t},\delta^+_{t}, \delta_t^-, \xi_t} & \sum_{\alpha \in A} \sum_{t \in T'}\left(\alpha\delta_{t \alpha }^{+}+(1-\alpha)\delta_{t \alpha }^{-}\right) & \\
& \qquad \qquad \qquad \qquad \qquad + \lambda_1\sum_{t \in T'}\gamma_{t \alpha } + \lambda_2\sum_{t \in T'}\xi_{t \alpha } & \nonumber \\
s.t. & \delta_{t}^{+}-\delta_{t \alpha }^{-}=y_{t}-q_{t \alpha }, & \qquad\forall t \in T',\forall \alpha \in A,\\
   & D^{1}_{t \alpha }=\frac{q_{\alpha t+1}-q_{\alpha t}}{x_{t+1}-x_{t}},
    & \qquad\forall t \in T',\forall \alpha \in A,\\   
 & D^{2}_{t \alpha }=\frac{\left(\frac{q_{\alpha t+1}-q_{\alpha t}}{x_{t+1}-x_{t}}\right)-\left(\frac{q_{\alpha t}-q_{\alpha t-1}}{x_{t}-x_{t-1}}\right)}{x_{t+1}-2x_{t} + x_{t-1}}.
  & \qquad\forall t \in T',\forall \alpha \in A,\\
 & \gamma_{t \alpha}\geq D^1_{t \alpha }, & \qquad\forall t \in T',\forall \alpha \in A,\\
  & \gamma_{t \alpha}\geq-D^1_{t \alpha}, & \qquad\forall t \in T',\forall \alpha \in A,\\
  & \xi_{t \alpha}\geq D^2_{t \alpha }, & \qquad\forall t \in T',\forall \alpha \in A,\\
 & \xi_{t \alpha}\geq-D^2_{t \alpha}, & \qquad\forall t \in T',\forall \alpha \in A,\\
 & \delta_{t \alpha}^{+},\delta_{t \alpha}^{-},\gamma_{t \alpha}, \xi_{t \alpha}\geq0, & \qquad\forall t \in T',\forall \alpha \in A,\\
  & q_{t \alpha} \leq q_{t \alpha'}, & \qquad \forall t \in T', \forall (\alpha, \alpha') \in A \times A, \alpha < \alpha',\nonumber \\  
  \end{eqnarray}


The output of our optimization problem is a sequence of ordered points $\{(x_t, q_{t \alpha})\}_{t \in T}$, for all $\alpha \in A$. The next step is to interpolate these points in order to provide an estimation for any other value of $x_t$. To address this issue, we propose using a linear interpolation, that will be developed in another study. Note that $q_{t \alpha}$ is a variable that represents only one point of the $\alpha$-quantile function $q_\alpha(x_t)$. 

The quantile estimation is done for different values of $\lambda$. By using different levels of penalization on the second difference, the estimation can be more or less adaptive to the fluctuation. It is important to notice that the usage of the $\ell_1$-norm as penalty leads to a piecewise linear solution $q_{t \alpha}$. % Referenciar?
Figure \ref{fig:npqar-results} shows the quantile estimation for a few different values of $\lambda$. 

	
% Para os gráficos antigos, usar a pasta /npqar/
\begin{figure}[htp]
  \centering
  \begin{minipage}[t]{0.4\linewidth}
    \centering
    \begin{minipage}[t]{\linewidth}
      \centering     \includegraphics[width=\textwidth]{Figuras/regressao-quantilica/icaraizinho-crossing-01}
      \subcaption{$\lambda_1 = 0, \, \lambda_2 = 0.1$}
    \end{minipage}
    \begin{minipage}[b]{\linewidth}
      \centering     \includegraphics[width=\textwidth]{Figuras/regressao-quantilica/icaraizinho-crossing-03}
      \subcaption{$\lambda_1 = 0, \, \lambda_2 = 0.3$}
    \end{minipage}
     \begin{minipage}[b]{\linewidth}
      \centering     \includegraphics[width=\textwidth]{Figuras/regressao-quantilica/icaraizinho-crossing-1}
      \subcaption{$\lambda_1 = 0, \, \lambda_2 = 1$}
     \end{minipage}
  \end{minipage}
  \begin{minipage}[t]{0.4\linewidth}
    \centering
    \begin{minipage}[t]{\linewidth}
      \centering     \includegraphics[width=\textwidth]{Figuras/regressao-quantilica/icaraizinho-crossing-3}
      \subcaption{$\lambda_1 = 0, \, \lambda_2 = 3$}
    \end{minipage}
    \begin{minipage}[b]{\linewidth}
      \centering     \includegraphics[width=\textwidth]{Figuras/regressao-quantilica/icaraizinho-crossing-10}
      \subcaption{$\lambda_1 = 0, \, \lambda_2 = 10$}
    \end{minipage}
     \begin{minipage}[b]{\linewidth}
      \centering     \includegraphics[width=\textwidth]{Figuras/regressao-quantilica/icaraizinho-crossing-200}
      \subcaption{$\lambda_1 = 0, \, \lambda_2 = 200$}
      \label{fig:npqar-cross}
     \end{minipage}
  \end{minipage}
  \caption{Quantile estimations for a few different values of $\lambda$. The quantiles represented here are $\alpha = (5\%, 10\%, 25\%, 50\%, 75\%, 90\%, 95\%)$. When $\lambda = 0.1$, on the upper left, we clarly see a overfitting on the estimations. The other extreme case is also shown, when $\lambda=200$ the nonparametric estimator converges to the linear model.}
  \label{fig:npqar-results}
\end{figure}

The first issue is how to select an appropriate value for $\lambda$. A simple way is to do it by inspection, which means to test many different values and pick the one that suits best our needs by looking at them. The other alternative is to use a metric to which we can select the best tune. We can achieve this by using a cross-validation method, for example.

The other issue occurs when we try to add more than one lag to the analysis at the same time. This happens because the problem solution is a set of points that we need to interpolate. This multivariate interpolation, however, is not easily solved, in the sense that we can either choose using a very naive estimator such as the K-nearest neighbors or just find another method that is not yet adopted for a wide range of applications.

%\subsection{Solar power data}
%
%While the Icaraizinho dataset has monthly observations for a wide range of time, we also test the same model for hourly data. In this case, we use the NP-QAR for solar power data. This dataset was retrieved from \textit{https://www.renewables.ninja/} and includes predicted hourly power data for the city of Tubarão (Brazil).
%This location was chosen because it is the spot of the biggest solar power plant in Brazil. 
%
%
%%Tabocas do Brejo Velho (usina solar "Horizonte" em construção, será a maior da america latina; -12.692, -44.006)
%
%
%%São José de Mipibu (usina solar  http://oportaln10.com.br/grupo-chines-instalara-fabrica-de-placas-solares-no-rn-50324/, -6.068 , -35.241)
%
%% Tubarão -28.467_-49.005_
%
%As solar power production is irradiation dependent, the best single predictor we may have is the hour of the day, as can be seen in the boxplot shown in Figure \ref{fig:solar-tubarao}
%
%\begin{figure}[h]
%	\centering
%	\includegraphics[width=0.8\linewidth]{Figuras/npqar-solar/hourly-solar-power.pdf}
%	\caption{Icaraizinho yearly data. Each serie consists of monthly observations for each year.}
%	\label{fig:solar-tubarao}
%\end{figure}
%
%
%
%
%
%\begin{figure}[h]
%	\centering
%		\includegraphics[width=0.8\linewidth]{Figuras/npqar-solar/com-divisao-lambda100.png}
%	\caption{Icaraizinho yearly data. Each serie consists of monthly observations for each year.}
%	\label{fig:npqar-solar-tubarao}
%\end{figure}



\subsection{A comparison between both approaches}

The last two sections introduced two different strategies to arrive in a Quantile Function $Q_{y_t|X}$. But what are the differences between using one method or the other? 

To provide a comparison between both approaches, we estimate a quantile function to predict the one-step ahead quantile function. We use as explanatory variable only the last observation $y_{t-1}$ - so $x_t = y_{t-1}$ - and estimate $\hat{q}_\alpha(y_{t-1})$, for every $\alpha \in \{0.05, 0.1, \dots, 0.9, 0.95 \}$. The result of both methods is shown on Figure \ref{fig:scatterplot-alphaquantiles}.

While the linear model produces $\alpha$-quantile functions which are linear by imposition, on the nonparametric model the $\alpha$-quantiles are flexible enough to form a hull on the data and adapt to its nonlinearities. The difference between the estimated quantile functions $\hat{Q}_{y_t|y_{t-1}}$ on both methods are shown on Figure \ref{fig:scatterplot-alphaquantiles}.

It is also important to test how the choice of the set $A$ affects the estimated quantile function. We experimented with two different sizes of $A$. In one of them, a dense grid of probabilities is used:   $A=\{0.005, 0.01, \dots, 0.99, 0.995 \}$, consisting of 199 elements. On the other only 19 elements are used to produce the quantile function ($A=\{0.05, 0.1, \dots, 0.9, 0.95 \}$).

\begin{figure}
  \centering
  \begin{minipage}[t]{\linewidth}
    \centering
    \begin{minipage}[t]{0.45\linewidth}
      \centering     \includegraphics[width=\textwidth]{Figuras/regressao-quantilica/quantile-linear-scatter}
    \end{minipage}
    \begin{minipage}[t]{0.45\linewidth}
      \centering     \includegraphics[width=\textwidth]{Figuras/regressao-quantilica/quantile-nonpar-scatter}
    \end{minipage}
  \end{minipage}
  \caption{Estimated $\alpha$-quantiles. On the left using a linear model and using a nonparametric approach (using $\lambda = 3$) on the right.}
  \label{fig:scatterplot-alphaquantiles}
\end{figure}




\begin{figure}
  \centering
  \begin{minipage}[t]{\linewidth}
    \centering
    \begin{minipage}[t]{0.45\linewidth}
      \centering     \includegraphics[width=\textwidth]{Figuras/regressao-quantilica/quantile-linear}
    \end{minipage}
    \begin{minipage}[t]{0.45\linewidth}
      \centering     \includegraphics[width=\textwidth]{Figuras/regressao-quantilica/quantile-nonpar}
    \end{minipage}
  \end{minipage}
  \caption{Estimated quantile functions, for different values of $y_{t-1}$. On the left using a linear model and using a nonparametric approach on the right.}
  \label{fig:quantiles-vs-xt}
\end{figure}

\begin{figure}
	\centering
	\begin{minipage}[t]{\linewidth}
		\centering
		\begin{minipage}[t]{0.45\linewidth}
			\centering     \includegraphics[width=\textwidth]{Figuras/regressao-quantilica/quantile-vs-alphas-linear}
		\end{minipage}
		\begin{minipage}[t]{0.45\linewidth}
			\centering     \includegraphics[width=\textwidth]{Figuras/regressao-quantilica/quantile-vs-alphas-nonpar}
		\end{minipage}
	\end{minipage}
	\caption{Sensitivity to different choices of set $A$. On the left, we have the estimated quantiles for the linear model, while on the right for the nonparametric model. On both, the red line shows the quantile function estimated with $A=\{0.005, 0.01, \dots, 0.99, 0.995 \}$, consisting of 199 elements. The blue line is the estimated quantile function when $A=\{0.05, 0.1, \dots, 0.9, 0.95 \}$, consisting of only 19 elements.}
	\label{fig:quantiles-vs-xt}
\end{figure}
\subsection{Overview}\label{overview}

\begin{itemize}
\item
  Let a time series be given by
  \[y_t = A(L)y_t + \beta x_t + \varepsilon_t,\] where the distribution
  of \(\varepsilon_t\) is unknown. In this case, the usage of a
  parametric model is hardly atrrative.
\item
  Quantile regression is one technique available to model this time
  series dynamics, by estimating a phin grid of \(\alpha\)-quantiles at
  once and forming a data-driven conditional distribution.
\item
  We explore different strategies of estimating the Conditional Quantile
  Regression focused on approaching the conditional distribution or
  \(y_{t+h|t}\), for a given horizon \(h\).
\end{itemize}

\section{Linear Models}\label{linear-models}

\subsection{Linear Models -
Formulation}\label{linear-models---formulation}

\tiny

\begin{eqnarray}
\min_{\beta_{0\alpha},\beta_\alpha,\varepsilon_{t\alpha}^{+}, \varepsilon_{t\alpha}^{-}} &  \sum_{\alpha \in A} \sum_{t \in T}\left(\alpha \varepsilon_{t \alpha}^{+}+(1-\alpha)\varepsilon_{t \alpha}^{-}\right) & \label{eq:linear-opt-1} \\
\mbox{s.t. } & \varepsilon_{t \alpha}^{+}-\varepsilon_{t \alpha}^{-}=y_{t} - \beta_{0\alpha} - \beta_{\alpha}^T x_{t}, & \qquad\forall t \in T,\forall \alpha \in A,\\
& \varepsilon_{t\alpha}^+,\varepsilon_{t\alpha}^- \geq 0, & \qquad\forall t \in T,\forall \alpha \in A,\\ \label{eq:linear-opt-ult}
& \beta_{0\alpha} + \beta_{\alpha}^T x_{t} \leq \beta_{0\alpha'} + \beta_{\alpha'}^T x_{t}, & \qquad \forall t \in T, \forall (\alpha, \alpha') \in A \times A,  \alpha < \alpha', \nonumber\\
\end{eqnarray}

\subsection{Linear Models - Resume}\label{linear-models---resume}

\begin{itemize}
\item
  As there are many explanatory variables for \(y_t\), it is interesting
  to do a regularization process in order to select only a subset to
  compose the model.
\item
  The next slides are going to cover a few different ways to achieve
  this using Mixed Integer Linear Programming.
\end{itemize}

\subsection{LM}\label{lm}

\{ \textbackslash{}begin\{IEEEeqnarray\}\{lr\}
\min\emph{\{\beta}\{0\alpha\},\beta\emph{\alpha,\varepsilon}\{t\alpha\}\^{}\{+\},
\varepsilon\emph{\{t\alpha\}\^{}\{-\}\} , \sum}\{\alpha \in A\}
\sum\emph{\{t \in T\}\left(\alpha \varepsilon}\{t
\alpha\}\^{}\{+\}+(1-\alpha)\varepsilon\emph{\{t
\alpha\}\^{}\{-\}\right) \span \label{eq:linear-opt-1}\textbackslash{}
\mbox{subject to} \span \nonumber \textbackslash{} \varepsilon}\{t
\alpha\}\^{}\{+\}-\varepsilon\emph{\{t \alpha\}\^{}\{-\}=y}\{t\} -
\beta\emph{\{0\alpha\} - \beta}\{\alpha\}\^{}T x\_\{t\}, \& \forall t
\in T,\qquad \forall \alpha \in A, \textbackslash{}
\varepsilon\emph{\{t\alpha\}\^{}+,\varepsilon}\{t\alpha\}\^{}- \geq 0,
\& \forall t \in T,\forall \alpha \in A,\textbackslash{}
\beta\emph{\{0\alpha\} + \beta}\{\alpha\}\^{}T x\_\{t\}
\leq \beta\_\{0\alpha`\} + \beta\emph{\{\alpha'\}\^{}T x}\{t\},
\span \nonumber \textbackslash{} \span \label{eq:linear-opt-ult}
\forall t \in T, \forall (\alpha, \alpha') \in A \times A,
\alpha \textless{} \alpha', \textbackslash{}end\{IEEEeqnarray\} \}

\subsection{Regularization by MILP -
Resume}\label{regularization-by-milp---resume}

\begin{itemize}
\item
  MILP models allow only \(K\) variables to be used for each
  \(\alpha\)-quantile. This means that only \(K\) coefficients
  \(\beta_{p\alpha}\) may have nonzero values, for each
  \(\alpha\)-quantile. It must be guaranteed by constraints on the
  optimization problem.
\item
  We present three forms of grouping probabilities while selecting
  variables
\end{itemize}

\subsection{\texorpdfstring{One model for each \(\alpha\)-quantile -
Formulation}{One model for each \textbackslash{}alpha-quantile - Formulation}}\label{one-model-for-each-alpha-quantile---formulation}

\tiny

\begin{eqnarray}
 \underset{\beta_{0\alpha},\beta_\alpha,z_{p \alpha} \varepsilon_{t \alpha}^{+},\varepsilon_{t \alpha}^{-}}{\text{min}} & \sum_{\alpha \in A} \sum_{t\in T}\left(\alpha\varepsilon_{t \alpha}^{+}+(1-\alpha)\varepsilon_{t\alpha}^{-}\right) \label{eq:mip0} \\
\mbox{s.t } & \varepsilon_{t \alpha}^{+}-\varepsilon_{t \alpha}^{-}=y_{t}-\beta_{0 \alpha}-\sum_{p=1}^{P}\beta_{p \alpha}x_{t,p},& \qquad\forall t \in T ,\forall \alpha \in A, \label{eq:mip1}\\
& \varepsilon_{t \alpha}^{+},\varepsilon_{t \alpha}^{-}\geq0,&\qquad\forall t \in T ,\forall \alpha \in A, \label{eq:mip2}\\
& - M z_{p \alpha} \leq \beta_{p \alpha} \leq M z_{p \alpha},&\qquad \forall \alpha \in A, \forall p\in P, \label{eq:mip3}\\
& \sum_{p=1}^P z_{p \alpha} \leq K, & \qquad \forall \alpha \in A, \label{eq:mip4}\\
& z_{p \alpha} \in \{0,1\},&\qquad \forall \alpha \in A, \forall p\in P, \label{eq:mip5}\\
& \beta_{0\alpha} + \beta_{\alpha}^T x_{t} \leq \beta_{0\alpha'} + \beta_{\alpha'}^T x_{t}, & \qquad \forall t \in T, \forall (\alpha, \alpha') \in A \times A,  \alpha < \alpha',\nonumber\\ \label{eq:mip6}
\end{eqnarray}

\subsection{\texorpdfstring{Defining groups for \(\alpha\)-quantiles -
Formulation}{Defining groups for \textbackslash{}alpha-quantiles - Formulation}}\label{defining-groups-for-alpha-quantiles---formulation}

\tiny

\begin{eqnarray}
 \underset{\beta_{0\alpha},\beta_\alpha,z_{p \alpha} \varepsilon_{t \alpha}^{+},\varepsilon_{t \alpha}^{-}}{\text{min}} & \sum_{\alpha \in A} \sum_{t\in T}\left(\alpha\varepsilon_{t \alpha}^{+}+(1-\alpha)\varepsilon_{t\alpha}^{-}\right) \label{eq:mip0} \\
\mbox{s.t } & \varepsilon_{t \alpha}^{+}-\varepsilon_{t \alpha}^{-}=y_{t}-\beta_{0 \alpha}-\sum_{p=1}^{P}\beta_{p \alpha}x_{t,p},& \qquad\forall t \in T ,\forall \alpha \in A, \label{eq:mip1}\\
& \varepsilon_{t \alpha}^{+},\varepsilon_{t \alpha}^{-}\geq0,&\qquad\forall t \in T ,\forall \alpha \in A, \label{eq:mip2}\\
& - M z_{p \alpha g} \leq \beta_{p \alpha} \leq M z_{p \alpha g},&\qquad \forall \alpha \in A, \forall p\in P, \forall g \in G \label{eq:mip3}\\
&z_{p \alpha g} := 2 - ( 1-z_{pg}) - I_{g\alpha}& \label{mipgrupzpa} \\
& \sum_{p=1}^P z_{p g} \leq K, & \qquad \forall g \in G, \label{eq:mip4}\\
& \beta_{0\alpha} + \beta_{\alpha}^T x_{t} \leq \beta_{0\alpha'} + \beta_{\alpha'}^T x_{t}, & \qquad \forall t \in T, \forall (\alpha, \alpha') \in A \times A,  \alpha < \alpha',\nonumber\\ \label{eq:mip6} \\
& \sum\limits_{g \in G} I_{g\alpha} = 1, & \forall \alpha \in A,\label{eq:mipgrupa} \\
& I_{g\alpha}, z_{pg} \in \{0,1\},& \forall p \in P, \quad \forall g \in G, 
\end{eqnarray}

\subsection{Defining groups by the introduction of switching variable -
Formulation}\label{defining-groups-by-the-introduction-of-switching-variable---formulation}

\tiny

\begin{eqnarray}
\underset{\beta_{0\alpha},\beta_\alpha,z_{p \alpha} \varepsilon_{t \alpha}^{+},\varepsilon_{t \alpha}^{-}}{\text{min}} & \sum_{\alpha \in A} \sum_{t\in T}\left(\alpha\varepsilon_{t \alpha}^{+}+(1-\alpha)\varepsilon_{t\alpha}^{-}\right) \label{eq:mipgr0} \\
\mbox{s.t } & \varepsilon_{t \alpha}^{+}-\varepsilon_{t \alpha}^{-}=y_{t}-\beta_{0 \alpha}-\sum_{p=1}^{P}\beta_{p \alpha}x_{t,p},& \qquad\forall t \in T ,\forall \alpha \in A, \label{eq:mipgr1}\\
& \varepsilon_{t \alpha}^{+},\varepsilon_{t \alpha}^{-}\geq0,&\qquad\forall t \in T ,\forall \alpha \in A, \label{eq:mipgr2}\\
& - M z_{p \alpha} \leq \beta_{p \alpha} \leq M z_{p \alpha},&\qquad \forall \alpha \in A, \forall p\in\{1,\dots,P\}, \label{eq:mipgr3}\\
& \sum_{p=1}^P z_{p \alpha} \leq K, & \qquad \forall \alpha \in A, \label{eq:mipgr4}\\
& z_{p \alpha} \in \{0,1\},&\qquad \forall \alpha \in A, \forall p\in\{1,\dots,P\}, \label{eq:mipgr5}\\
& \beta_{0\alpha} + \beta_{\alpha}^T x_{t} \leq \beta_{0\alpha'} + \beta_{\alpha'}^T x_{t}, & \qquad \forall t \in T, \forall (\alpha, \alpha') \in A \times A,  \alpha < \alpha',\nonumber\\ \label{eq:mipgr6} \\
& z_{p\alpha} - z_{p\alpha+1} \leq m_{p\alpha}, & \qquad \forall \alpha \in A', \qquad \forall p \in P  \\
& \sum_{\alpha \in A'} r_\alpha \leq |G| - 1 
\label{eq:mipgr} \\
\end{eqnarray}

where \(A' = A\setminus \{|A|\}\) \normalsize

\section{Nonparametric model}\label{nonparametric-model}

\subsection{Nonparametric model -
Formulation}\label{nonparametric-model---formulation}

\tiny

\begin{eqnarray}
\min_{q_{\alpha t},\delta^+_{t}, \delta_t^-, \xi_t} & \sum_{\alpha \in A} \sum_{t \in T'}\left(\alpha\delta_{t \alpha }^{+}+(1-\alpha)\delta_{t \alpha }^{-}\right) & \\
& \qquad \qquad \qquad \qquad \qquad + \lambda_1\sum_{t \in T'}\gamma_{t \alpha } + \lambda_2\sum_{t \in T'}\xi_{t \alpha } & \nonumber \\
s.t. & \delta_{t}^{+}-\delta_{t \alpha }^{-}=y_{t}-q_{t \alpha }, & \qquad\forall t \in T',\forall \alpha \in A,\\
   & D^{1}_{t \alpha }=\frac{q_{\alpha t+1}-q_{\alpha t}}{x_{t+1}-x_{t}},
    & \qquad\forall t \in T',\forall \alpha \in A,\\   
 & D^{2}_{t \alpha }=\frac{\left(\frac{q_{\alpha t+1}-q_{\alpha t}}{x_{t+1}-x_{t}}\right)-\left(\frac{q_{\alpha t}-q_{\alpha t-1}}{x_{t}-x_{t-1}}\right)}{x_{t+1}-2x_{t} + x_{t-1}}.
  & \qquad\forall t \in T',\forall \alpha \in A,\\
 & \gamma_{t \alpha}\geq D^1_{t \alpha }, & \qquad\forall t \in T',\forall \alpha \in A,\\
  & \gamma_{t \alpha}\geq-D^1_{t \alpha}, & \qquad\forall t \in T',\forall \alpha \in A,\\
  & \xi_{t \alpha}\geq D^2_{t \alpha }, & \qquad\forall t \in T',\forall \alpha \in A,\\
 & \xi_{t \alpha}\geq-D^2_{t \alpha}, & \qquad\forall t \in T',\forall \alpha \in A,\\
 & \delta_{t \alpha}^{+},\delta_{t \alpha}^{-},\gamma_{t \alpha}, \xi_{t \alpha}\geq0, & \qquad\forall t \in T',\forall \alpha \in A,\\
  & q_{t \alpha} \leq q_{t \alpha'}, & \qquad \forall t \in T', \forall (\alpha, \alpha') \in A \times A, \alpha < \alpha',\nonumber \\  
  \end{eqnarray}

\subsection{Nonparametric vs.~Linear
Model}\label{nonparametric-vs.linear-model}

\begin{itemize}
\tightlist
\item
  The nonparametric approach is more flexible to capture
  heteroscedasticity.
\end{itemize}

\begin{figure}
  \centering
  \begin{minipage}[t]{\linewidth}
    \centering
    \begin{minipage}[t]{0.45\linewidth}
      \centering     \includegraphics[width=\textwidth]{../Figuras/regressao-quantilica/icaraizinho-quantile-linear}
    \end{minipage}
    \begin{minipage}[t]{0.45\linewidth}
      \centering     \includegraphics[width=\textwidth]{../Figuras/regressao-quantilica/icaraizinho-quantile-nonpar-lambda30}
    \end{minipage}
  \end{minipage}
  \caption{Estimated quantile functions, for different values of $y_{t-1}$. On the left using a linear model and using a nonparametric approach on the right.}
  \label{fig:quantiles-vs-xt}
\end{figure}

\subsection{Nonparametric vs.~Linear
Model}\label{nonparametric-vs.linear-model-1}

\begin{itemize}
\tightlist
\item
  This flexibility might lead to overfitting, if we don't select a
  proper penalty, as shown below:

  \begin{figure}
          \centering     \includegraphics[width=0.6 \textwidth]{../Figuras/regressao-quantilica/icaraizinho-crossing-01}
  \caption{Example of a overfitted quantile function}
  \label{fig:convergence}
  \end{figure}
\end{itemize}

\section{Goals}\label{goals}

\subsection{Goals}\label{goals-1}

Goals of this work are:

\begin{itemize}
\item
  Using multiple quantiles to estimate the empirical conditional
  distribution of variables in a time series
\item
  Producing a model identification methodology
\item
  Natural Ressources and Financial applications
\end{itemize}
